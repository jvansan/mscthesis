% !TEX root = diss.tex

\chapter{Conclusions}

\begin{doublespace}

HAT reactions are amongst the simplest radical chemical transformations. This
can be deceiving, as there are many factors that influence HAT reactions that
remain poorly understood. It is important to develop a full understanding of HAT
reactions as they are a fundamental step in many biochemical processes. In this
thesis, three aspects of HAT reactivity were explored utilizing quantum chemical
techniques.

First, the role of non-covalent binding in the pre-reaction complex of HAT
reactions was investigated. In particular, the relationship between the
calculated pre-reaction complex binding energies and experimentally determined
Arrhenius pre-exponential factors (A-factors) was examined for a series of
thermoneutral or nearly thermoneutral reactions involving the formation and
destruction of oxygen-centred radicals. The interpretation of the results of
this investigation relies on an assumption: the mechanism for formal HAT, which
is normally described as either direct HAT of PCET, exists on a continuum that
is described by Equation~\ref{eq:A-theory}.

It was demonstrated that there may be a correlation between A-factors and
pre-reaction complex binding energies for (nearly) thermoneutral HAT reactions
given that the reactions follow similar reaction mechanisms. This is an
important caveat, as it is clear that binding energies do not serve as a
diagnostic for fully describing all the entropic contributions to A-factors. In
order to apply this analysis to further systems, future work should aim at
determining a quantitative diagnostic for the contributions of PCET and direct
HAT to the overall hydrogen transfer reaction mechanism.

Next, the validity of the Bell-Evans-Polanyi (BEP) principle was investigated by
analyzing a series of HAT reactions in which a hydrogen atom is abstracted from
a carbon centre by the \cumo\ radical. A hypothesis on the basis of a
group-additivity scheme posed by our experimental colleagues stated that if the
BEP principle was valid, there should exist two linear relationships for
\ch{C-H} bonds. Namely, one in which the incipient radical is delocalized into a
$\pi$-system (benzylic or allylic), and the other in which the remaining alkyl
radicals are largely localized. Detailed analysis of experimentally determined
$\log(k_H/n)$ plotted against theoretically determined \ch{C-H} BDEs
demonstrated that there was reasonable correlation in the case of
allylic/benzylic substrates, however the relationship did not hold for all other
alkyl substrates. Breaking the larger group of alkyl substrates into smaller
chemical groups did not resolve any simple linear relationships. This may
support the notion that the BEP principle in not a valid LFER.

To provide further evidence, calculations were performed to determine the
structures of relevant transition state complexes and reaction barrier heights.
It was demonstrated that HAT reactions involving \cumo\ likely proceed through a
mechanism that is dominated by direct HAT. As a result differences in mechanism
may be ruled as a factor. Upon decomposition of the free energy barrier heights,
it was revealed that HAT reactions involving \cumo\ are entropy-controlled, and
non-isoentropic. It is established in the literature that entropy-controlled
processes do not follow LFERs consistently.\cite{Exner1973} Therefore, while
these results do not invalidate the BEP principle as a LFER, it is apparent that
the model is an over-simplification of the complexity associated with HAT
reactions involving \cumo. This is likely to be the case for many HAT reactions.
As a result, the BEP principle should not be used to as a quantitative
prediction tool, but should remain a conceptual framework to qualitatively
describe changes in reaction rates.

Finally, recent experimental evidence demonstrated that non-redox active metal
cations can have an inhibitory effect on HAT reactions involving oxygen-centred
radicals. As these metals are found ubiquitously in biological systems, it was
suggested that this may be a form of chemo-protection against radical induced
oxidation of biomaterials, in particular proteins. It was previously reported
that mechanism for this inhibition relies on \ch{C-H} bond deactivation, which
may be the result of a metal cation binding to a substrate. Herein, I sought to
determine the exact mechanism by which this occurs. It was previously
hypothesized that the metal bound to the substrate via an electron rich centre,
such as the oxygen of a carbonyl. Then, electron density is withdrawn from a
neighbouring \ch{C-H} $\sigma^*$ orbital, which is normally populated due to
hyperconjugation. Withdrawing electron density from the \ch{C-H} $\sigma^*$
orbital strengthens the bond, and thus the HAT reaction barrier height may
increase, as per the BEP principle.

Two amides, DMA and DIA, were used as small models for protein systems. The
calculated \ch{C-H} BDEs of these substrates both with and without metal cations
demonstrate that specific metal-substrate interactions can increase the
effective \ch{C-H} BDEs in the parent molecule. This is however complicated by
the fact that the product radical can also interact with the metal cation, which
can either increase or decrease the BDE. Thus, BDEs may not properly reflect the
effect metal cations have on HAT barrier heights.

The hypothesis that \ch{C-H} bond strengths are increased by interactions of the
metal cation with the substrate also precludes any notion of interactions of
that metal with other compounds. This became abundantly obvious in calculating
the reaction barrier heights including \ch{NaCl} of DMA and DIA with the
radicals \cumo\ and \bno. Specifically, the predicted changes in barrier heights
as a result of binding of \ch{NaCl} to the substrate were found to both increase
and decrease. Analysis of the interaction of \ch{NaCl} with both the radical and
extensions of the substrate demonstrated that secondary interaction can either
stabilize or destabilize the TS complex. The aforementioned hypothesis does not
account for secondary interactions. Therefore, the results obtained herein do
not support the hypothesis.

Furthermore, the complexity protein systems increases the likelihood that alkali
and alkaline earth metal cations will interact not only with a single peptide
moiety in a protein environment. As a consequence, it is unlikely that these
metal cations serve as natural chemo-protective agents against radical induced
oxidation in biological systems.

There is one important theme that arises from all the results in this thesis:
HAT reactions are deceptively simple. In each chapter, an attempt was made to
make a generalization about the physico-chemical properties of HAT reactions. In
each case, some properties of HAT reactivity were elucidated, however further
questions arose. This work sets the stage for further research into three
important aspects as HAT reactivity. 


\end{doublespace}

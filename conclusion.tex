% !TEX root = diss.tex

\chapter{Conclusions}

\begin{doublespace}

HAT reactions are amongst the simplest radical chemical transformations. This
can be deceiving, as there are many poorly understood factors that influence
HAT. It is important to develop a full understanding of HAT reactions as they
are a fundamental step in many biochemical processes. The radical-induced
oxidation of biomaterials is often trigged by HAT reactions, and has been
implicated in a number of degenerative disease states. In this thesis, three
aspects of HAT reactivity were explored using quantum chemical techniques.

First, the role of non-covalent binding in the pre-reaction complex of HAT
reactions was investigated. In particular, the relationship between the
calculated pre-reaction complex binding energies and experimentally determined
Arrhenius pre-exponential factors (A-factors) was examined for a series of
thermoneutral or nearly thermoneutral reactions involving the formation and
destruction of oxygen-centred radicals. The interpretation of the results of
this investigation relies on an assumption: the mechanism for formal HAT, which
is normally described as either direct HAT of PCET, exists on a continuum that
is described by Equation~\ref{eq:A-theory}.

It was demonstrated that there may be a correlation between A-factors and
pre-reaction complex binding energies for (nearly) thermoneutral HAT reactions
given that the reactions follow similar reaction mechanisms. This is an
important caveat, as it is clear that binding energies do not serve as a
diagnostic for fully describing all the entropic contributions to A-factors.
Specifically, for a set of ten self-exchange and pseudo-self-exchange reaction,
six out of ten of the reactions studied share similar reaction mechanisms, and
there is a strong correlation ($R^2$ = 0.949) of pre-reaction complex binding
energies with A-factors. In order to apply this analysis to further systems,
future work should aim at determining a quantitative diagnostic for the
contributions of PCET and direct HAT to the overall hydrogen transfer reaction
mechanism.

Next, the validity of the Bell-Evans-Polanyi (BEP) principle was investigated by
analyzing a series of HAT reactions in which a hydrogen atom is abstracted from
a carbon centre by the \cumo\ radical. A hypothesis on the basis of
group-additivity states that if the BEP principle is valid, there should exist
two linear relationships for \ch{C-H} bonds, namely, one in which the incipient
radical is delocalized into a $\pi$-system (benzylic or allylic), and the other
in which the remaining alkyl radicals are largely localized. Detailed analysis
of experimentally determined $\log(k_H/n)$ plotted against theoretically
determined \ch{C-H} BDEs demonstrated that there was reasonable correlation
($R^2$ = 0.889) in the case of allylic/benzylic substrates, however the
correlation was not strong for all other alkyl substrates ($R^2$ = 0.641).
Breaking the larger group of alkyl substrates into smaller chemical groups
suggested that specific groups of alkyl containing species (e.g. cyclic alkanes
and alkyl group with heteroatomic neighbours) may demonstrate linear
relationships, however more data are needed to support this.

To explore this further, calculations were performed to determine the structures
of relevant transition state complexes and reaction barrier heights. It was
demonstrated that HAT reactions involving \cumo\ likely proceed through a
mechanism that is dominated by direct HAT. As a result differences in
mechanistic details may be ruled as a factor contributing to the observed poor
correlation. Decomposition of the free energy barrier heights revealed that HAT
reactions involving \cumo\ are entropy-controlled ($-T\Delta S^\ddagger >>
\Delta H^\ddagger$), and non-isoentropic. It is established in the literature
that entropy-controlled processes do not follow LFERs
consistently.\cite{Exner1973} Therefore, while these results do not invalidate
the BEP principle as a LFER, it is apparent that the model is an
over-simplification of the complexity associated with HAT reactions involving
\cumo. This is likely to be the case for many HAT reactions. As a result, the
BEP principle should not be used to as a quantitative prediction tool, but
should remain a conceptual framework to qualitatively describe changes in
reaction rates.

Finally, recent experimental evidence demonstrated that non-redox active metal
cations can have an inhibitory effect on HAT reactions involving oxygen-centred
radicals. As these metals are found ubiquitously in biological systems, it was
suggested that this may be a form of chemo-protection against radical induced
oxidation of biomaterials, in particular proteins. It was previously reported
that the mechanism for this inhibition relies on \ch{C-H} bond deactivation,
which may be the result of a metal cation binding to a substrate. Herein, I
sought to determine the exact mechanism by which this occurs. It was previously
hypothesized that the metal bound to the substrate via an electron rich centre,
such as the oxygen of a carbonyl. Then, electron density is withdrawn from a
neighbouring \ch{C-H} $\sigma^*$ orbital, which is normally populated through
hyperconjugation. Withdrawing electron density from the \ch{C-H} $\sigma^*$
orbital strengthens the bond, and thus the HAT reaction barrier height may
increase, as per the BEP principle.

Two amides, DMA and DIA, were used as small models for protein systems. The
calculated \ch{C-H} BDEs of these substrates both with and without metal cations
demonstrate that specific metal-substrate interactions can increase the
effective \ch{C-H} BDEs in the parent molecule. The calculated $N$-methyl
\ch{C-H} BDEs of DMA increase by an average of 1.4 \kcalmol\ upon complexation
of \ch{NaCl}. This is however complicated by the fact that the product radical
can also interact with the metal cation, which can either increase or decrease
the BDE. For example, the acetyl \ch{C-H} BDE of DMA decreases by 0.1 \kcalmol\
upon complexation of \ch{NaCl}. As a result of metal-radical interaction, BDEs
may not properly reflect the effect metal cations have on HAT barrier heights.

The hypothesis that \ch{C-H} bond strengths are increased by interactions of the
metal cation with the substrate does not account for interactions of that metal
with species other than the substrate. This became abundantly obvious in
calculating the reaction barrier heights including \ch{NaCl} of DMA and DIA with
the radicals \cumo\ and \bno. Specifically, the predicted changes in barrier
heights as a result of binding of \ch{NaCl} to the substrate were found to both
increase and decrease. For the HAT reaction of DMA and \cumo\ including
\ch{NaCl}, the predicted reaction barrier height for abstraction from an
$N$-methyl hydrogen trans to the carbonyl increases as there are no interactions
of \ch{NaCl} with \cumo. However, for abstraction from an $N$-methyl cis to the
carbonyl decrease as a result of a charge transfer interaction between \ch{NaCl}
and \cumo. Analysis of the interactions of \ch{NaCl} with both the radical and
extensions of the substrate demonstrated that secondary interaction can either
stabilize or destabilize the TS complex.

The aforementioned hypothesis does not account for secondary interactions,
however it was based on experimental results which demonstrate that alkali and
alkaline earth metal salts do inhibit HAT reactions of amides with \cumo. The
results obtained herein do not support the hypothesis. An explanation for this
discrepancy remains unclear, however I have speculated that this may be a result
of problems with the model chosen. Specifically, the experimental data suggests
that stoichiometric ratios of substrates and metal cations are somehow related
to the inhibitory effect. No multiple coordinating stoichiometries were
considered herein.

Furthermore, the complexity of protein systems increases the likelihood that
alkali and alkaline earth metal cations will interact not only with a single
peptide moiety in a protein environment. On the basis of these limited results,
it is possible that these metal cations may not serve as natural
chemo-protective agents against radical induced oxidation in biological systems.

There is one important theme that arises from all the results in this thesis:
HAT reactions appear deceptively simple. In each chapter, an attempt was made to
make a generalization about the physico-chemical properties of HAT reactions. In
each case, some properties of HAT reactivity were elucidated, while further
questions arose. This work sets the stage for further research into three
important aspects as HAT reactivity.


\end{doublespace}

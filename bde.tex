% !TEX root = diss.tex

\chapter{Interrogation of the Bell-Evans-Polanyi Principle: Investigation of the
Bond Dissociation Enthalpies correlated with Hydrogen Atom Transfer Rate Constants}
\label{ch:bde}

\jnote{Experimental BDEs from Bordwell\cite{Bordwell1988} cycle are possible unreliable in the case where PCET occurs \cite{Miller2016}}

Bond dissociation enthalpies (BDEs) are one of the most fundamental properties used by chemists to understand chemical reactions. This is especially apparent in the case of related HAT reactions, where the hydrogen atom acceptor is constant in all cases. Then, the difference in enthalpy of formation will depend entirely on the BDE.  Moreover, BDEs are measurable using a large number of different experimental techniques. The \emph{de facto} reference for BDEs is \cite{Luo2002} CRC Handbook of Bond Dissociation Enthalpies.

\section{Methods}

Experimental rate constants were have either provided from unpublised results from our colleagues, the Bietti group in Rome, or come from literature sources.\cite{Bietti2010, Bietti2011, Pischel2001, Salamone2011, Salamone2012, Salamone2012a, Salamone2013, Salamone2015} All rate constants come from laser flash photolysis (LFP) experiments of \cumo~ with the substrates of interest. Acetonitrile solvent and ambient conditions (298 K and 1 atm) were used in all cases. For those results which have are unpublished, \cumo~ is generated by laser pulses at either 266 nm or 355 nm in solutions of excess dicumyl peroxide. Many of the literature results are also from the Bietti group, where the same procedure is used. Other results may have small variations in experimental details, however, all results are well time-resolved.

Observed rate constants ($k_{obs}$) are obtained from transient absorption decay traces of \cumo monitored at 485 nm. The observed rate constant is plotted against concentration of substrate to provide bimolecular HAT rate constants ($k_H$) as the slope ($k_{obs} = k_0 + k_H[substrate]$). The unimolecular decay rate constant for \cumo ($k_0$) in acetonitrile is on the order of 7.5 \E{5} s$^{-1}$.\cite{Avila1995}

All quantum chemical calculations were performed using the Gaussian 09 software package.\cite{Frisch2009} Several composite quantum chemical method which are implemented in Gaussian 09 were used in this work: W1BD, CBS-QB3 and the restricted open-shell variant ROCBS-QB3, CBS-APNO, and G4 and the MP2 variable G4(MP2). Each of these methods is briefly described below.

% and an approach using ROCCSD(T) with locally-dense basis sets (LDBS).
\jnote{I have omitted the LDBS method and results as it does not seem to add anything significant to the story.}

\subsection{Quantum chemical composite procedures}

\noindent \textbf{W1BD}

The highest accuracy method used is W1BD, which employs seven different calculations to obtain highly correlated electronic energies, as well as thermochemically corrected quantities. This method is very computationally expensive, and thus cannot be applied to the larger species of interest in this work. Geometries and thermochemical corrections come from DFT-based B3LYP calculations with nearly complete cc-pVTZ+d basis sets. A zero-point energy (ZPE) scaling factor of 0.985 is used for harmonic frequency calculations. The electronic energy comes from several additive corrections involving the Brueckner Doubles\cite{Barnes2009} (BD) variation of coupled cluster and various large basis sets extrapolated to the complete basis set limit. Corrections for core-electron correlation and relativistic contributions are computed using the uncontracted variate of the cc-pVTZ+2df basis sets, known as MTsmall.\cite{Martin1999}
\\

\noindent \textbf{CBS methods}

The Complete Basis Set (CBS) methods of Petersson and coworkers\cite{Montgomery1999, Montgomery2000, Ochterski1996, Wood2006} are widely used because of the relatively low computational cost (compared to other composite procedures), and well established accuracy.\cite{Somers2015, Simmie2015} CBS-QB3\cite{Montgomery1999, Montgomery2000} utilises DFT-based B3LYP optimisation and scaled (ZPE scaling factor = 0.990) frequencies with modified triple-zeta Pople style basis sets. Electronic energies are obtained by extrapolation of medium basis set CCSD(T) and MP4SDQ. Small empirical corrections for are added in an ad-hoc fashion to achieve more accurate results compared to the parametrisation sets.\cite{Petersson2001} ROCBS-QB3 is an identical procedure, except spin-resticted wave functions are are in place of unrestricted wave functions. This is done to eliminate spin contamination, and the use of an restricted open-shell definition has been shown to produce more accurate BDEs.\cite{DiLabio1999} CBS-QB3 has been implemented for first, second, and third row periods of elements.

Atomic pair natural orbital (APNO) expansions are a method used for averaging over multiple Slater determinants. The use of APNOs allows for small basis set extrapolation of higher order correlation energies to converge more rapidly to the complete basis set limit. This approach is used in the CBS-APNO method.\cite{Ochterski1996} Geometries and scaled (ZPE scaling factor = 0.989) frequencies are obtained at the QCISD/6-311G(d,p) level of theory. Similar to CBS-QB3, the extrapolation of moderate basis set MP4SDQ and QCISD(T) results gives the electronic energy. An empirical correction is also used in CBS-APNO. Even though CBS-APNO is more accurate, the expansion of APNOs makes CBS-APNO more computationally demanding than CBS-QB3. As a results, it has only been implemented for first and second row periods, and is thus less commonly used in literature.
\\

\noindent \textbf{G$n$ methods}

The Gaussian$-n$ (G$n$) series of methods originate from the Pople group,\cite{Pople1989} where G4 is the fourth generation. G4 utilises moderately large basis sets and extrapolation techniques with CCSD(T) calculations to obtained highly correlated electronic energies. G4(MP2) uses MP2 in place is CCSD(T) and is thus less computationally expensive, but also gives a less complete description of electron correlation. Both methods use the B3LYP/6-31(2df,p) level of theory for optimisation and frequency calculations with a ZPE scaling factor of 0.9854. G4 results have been described as generally on par with CBS-QB3 results,\cite{Somers2015, Simmie2015} but calculations are more computationally expensive.

\subsection{Transition state calculations}

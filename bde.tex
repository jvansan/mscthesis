% !TEX root = diss.tex

\chapter{Interrogation of the Bell-Evans-Polanyi Principle: Investigation of the
Bond Dissociation Enthalpies correlated with Hydrogen Atom Transfer Rate Constants}
\label{ch:bde}



The Bell-Evans-Polanyi (BEP) principle states that for two closely related reactions, the difference in activation energy is proportional to the difference in their enthalpy of reaction.\cite{Bell1936,Evans1938,Dill2003} This is commonly expressed as the linear free energy relationship in~\ref{eq:bep}. Initially, the BEP principle was used as a simple model to explain the Br{\o}nsted catalysis law, which states that the stronger and acid is, the faster the catalysed reaction will proceed.\cite{Bronsted1924} These relationships can be described schematically: the more stable the product, the lower the reaction barrier, as seen in~\ref{fig:bep}.

\begin{figure}[htb]
  \centering
  \includegraphics[width=0.7\textwidth]{figures/bep}
  \caption{Energy profiles for a series of related exothermic reactions illustrating the Bell-Evans-Polanyi Principle.}
\label{fig:bep}
\end{figure}

A modern use of the BEP principle is to estimate the activation energy of related reactions. If the BEP relationship holds for a series of related HAT reactions, then bond dissociation enthalpies (BDEs) should correlate with the activation energy. In this case, increased bond strengths would represent a destabilisation in the TS complex, and thus a decrease in reaction rate. In practice, plots of BDEs against the logarithm of rate constant are used. An interesting example of this is the work of \citet{Pratt2003}, in which the free radical oxidation of unsaturated lipids is examined. They achieve this through the correlation of theoretically determined C-H and \ch{C-OO^.} bond strengths with experimentally measured HAT rate constants and \ch{O2} addition rate constants, respectively. BEP plots (BDE vs. $\log k$) for a large range of polyunsaturated fatty acid models show good correlation for both the C-H bonds and \ch{C-OO^.} bonds examined. This demonstrates that BDEs have a direct impact on the reaction barrier height, giving validation to the BEP principle.

In addition to this, there is no prescription as to how broadly the BEP principle can be applied. In this work, I seek to explore this issue. In order to achieve this, I explore HAT reactions involving the abstraction of C-H bonds by \cumo, for which many rate constants have been published.\cite{Bietti2010, Bietti2011, Pischel2001, Salamone2011, Salamone2012, Salamone2012a, Salamone2013, Salamone2015} Additional unpublished rate constants have been provided by our experimental colleagues in Rome.

BDEs are measurable using a large number of different experimental techniques. The \emph{de facto} reference for BDEs is the \emph{CRC Handbook of Bond Dissociation Enthalpies}.\cite{Luo2002} Unfortunately, not all experimental methods give reliable BDE data. For example, BDEs from Bordwell\cite{Bordwell1988} thermochemical cycle are possibly unreliable in the case where PCET occurs.\cite{Miller2016} Therefore, quantum chemistry is a useful tool for studying BDEs, as it is facile to compute reliable BDEs. For example, an arbitrary \ch{X-H} bond strength is given by:

\begin{equation}
  \Delta H(BDE) = H(\ch{X-H}) - H(\ch{X^.}) - H(\ch{H^.})
\end{equation}

\noindent where $\Delta H(BDE)$ is the BDE, and the right-hand terms are the enthalpies of the substrate, the incipient radical, and the hydrogen atom, respectively.

DFT-based methods have been shown to give reliable relative BDEs, however, highly correlated methods are required to predict chemically accurate (sub-\kcalmol) BDEs.\cite{DiLabio1999, Chan2012, Wiberg2014} Often this is done using composite quantum chemical procedures. Unfortunately, due to the computational cost of some of these procedures, calculations are often limited to small molecules. Additionally, there is currently no literature which compares the ability of common composite methods to predict accurate BDEs. Therefore, another aim of the work is to determine which composite procedure most efficiently gives accurate BDEs.

\section{Methods}

Experimental rate constants were have either provided from unpublised results from our colleagues, the Bietti group in Rome, or come from literature sources.\cite{Bietti2010, Bietti2011, Pischel2001, Salamone2011, Salamone2012, Salamone2012a, Salamone2013, Salamone2015} All rate constants come from laser flash photolysis (LFP) experiments of \cumo~ with the substrates of interest. Acetonitrile solvent and ambient conditions (298 K and 1 atm) were used in all cases. For those results which have are unpublished, \cumo~ is generated by laser pulses at either 266 nm or 355 nm in solutions of excess dicumyl peroxide. Many of the literature results are also from the Bietti group, where the same procedure is used. Other results may have small variations in experimental details, however, all results are well time-resolved.

Observed rate constants ($k_{obs}$) are obtained from transient absorption decay traces of \cumo monitored at 485 nm. The observed rate constant is plotted against concentration of substrate to provide bimolecular HAT rate constants ($k_H$) as the slope ($k_{obs} = k_0 + k_H[substrate]$). The unimolecular decay rate constant for \cumo ($k_0$) in acetonitrile is on the order of 7.5 \E{5} s$^{-1}$.\cite{Avila1995}

All quantum chemical calculations were performed using the Gaussian 09 software package.\cite{Frisch2009} Several composite quantum chemical method which are implemented in Gaussian 09 were used in this work: W1BD, CBS-QB3 and the restricted open-shell variant ROCBS-QB3, CBS-APNO, and G4 and the MP2 variable G4(MP2). Each of these methods is briefly described below.

% and an approach using ROCCSD(T) with locally-dense basis sets (LDBS).
\jnote{I have omitted the LDBS method and results as it does not seem to add anything significant to the story.}

\subsection{Quantum chemical composite procedures}

\noindent \textbf{W1BD}

The highest accuracy method used is W1BD, which employs seven different calculations to obtain highly correlated electronic energies, as well as thermochemically corrected quantities. This method is very computationally expensive, and thus cannot be applied to the larger species of interest in this work. Geometries and thermochemical corrections come from DFT-based B3LYP calculations with nearly complete cc-pVTZ+d basis sets. A zero-point energy (ZPE) scaling factor of 0.985 is used for harmonic frequency calculations. The electronic energy comes from several additive corrections involving the Brueckner Doubles\cite{Barnes2009} (BD) variation of coupled cluster and various large basis sets extrapolated to the complete basis set limit. Corrections for core-electron correlation and relativistic contributions are computed using the uncontracted variate of the cc-pVTZ+2df basis sets, known as MTsmall.\cite{Martin1999}
\\

\noindent \textbf{CBS methods}

The Complete Basis Set (CBS) methods of Petersson and coworkers\cite{Montgomery1999, Montgomery2000, Ochterski1996, Wood2006} are widely used because of the relatively low computational cost (compared to other composite procedures), and well established accuracy.\cite{Somers2015, Simmie2015} CBS-QB3\cite{Montgomery1999, Montgomery2000} utilises DFT-based B3LYP optimisation and scaled (ZPE scaling factor = 0.990) frequencies with modified triple-zeta Pople style basis sets. Electronic energies are obtained by extrapolation of medium basis set CCSD(T) and MP4SDQ. Small empirical corrections for are added in an ad-hoc fashion to achieve more accurate results compared to the parametrisation sets.\cite{Petersson2001} ROCBS-QB3 is an identical procedure, except spin-resticted wave functions are are in place of unrestricted wave functions. This is done to eliminate spin contamination, and the use of an restricted open-shell definition has been shown to produce more accurate BDEs.\cite{DiLabio1999} CBS-QB3 has been implemented for first, second, and third row periods of elements.

Atomic pair natural orbital (APNO) expansions are a method used for averaging over multiple Slater determinants. The use of APNOs allows for small basis set extrapolation of higher order correlation energies to converge more rapidly to the complete basis set limit. This approach is used in the CBS-APNO method.\cite{Ochterski1996} Geometries and scaled (ZPE scaling factor = 0.989) frequencies are obtained at the QCISD/6-311G(d,p) level of theory. Similar to CBS-QB3, the extrapolation of moderate basis set MP4SDQ and QCISD(T) results gives the electronic energy. An empirical correction is also used in CBS-APNO. Even though CBS-APNO is more accurate, the expansion of APNOs makes CBS-APNO more computationally demanding than CBS-QB3. As a results, it has only been implemented for first and second row periods, and is thus less commonly used in literature.
\\

\noindent \textbf{G$n$ methods}

The Gaussian$-n$ (G$n$) series of methods originate from the Pople group,\cite{Pople1989} where G4 is the fourth generation. G4 utilises moderately large basis sets and extrapolation techniques with CCSD(T) calculations to obtained highly correlated electronic energies. G4(MP2) uses MP2 in place is CCSD(T) and is thus less computationally expensive, but also gives a less complete description of electron correlation. Both methods use the B3LYP/6-31(2df,p) level of theory for optimisation and frequency calculations with a ZPE scaling factor of 0.9854. G4 results have been described as generally on par with CBS-QB3 results,\cite{Somers2015, Simmie2015} but calculations are more computationally expensive.

\subsection{Transition state calculations}

\section{Comparison of composite method for the prediction of BDEs}

In order to determine the best method for BEP principle analysis, and to investigate which is the most accurate composite method, the BDEs of 49 species have been calculated. This set of species contains a wide variety of chemical functionalities, thus this set may be described as a comprehensive test of these methods for C-H BDEs. Given that W1BD is the most accurate method used, these results have been used for comparison to other composite method. Unfortunately, BDE for only 33 out of the 49 species studied were able to calculated by W1BD due to computational restrictions. Therefore, literature BDEs from \citet{Luo2002} for all species in the set are also used for comparison. The literature and calculated BDEs are listed in Appendix X, \jnote{TABLE REF}.

One of the most used tools for assessing the quality of computational methods is the \emph{mean absolute error} (MAE) with respect to benchmark values for a given data set.\cite{Savin2014} The MAE is calculated as

\begin{equation}
  \mathrm{MAE} = \frac{1}{N} \sum | E_{ref} - E_{calc}|
\end{equation}

\noindent where for a set of $N$ reference values, the MAE is the average of the mean differences of the reference energy ($E_{ref}$) and the calculated value ($E_{calc}$). The MAE with respect to W1BD and literature shall be reported herein as ``MAE$_{\mathrm{W1BD}}$ (MAE$_{\mathrm{Literature}}$)''. An additional semi-quantitative metric which I used to evaluate the accuracy of composite procedures to reproduce experimental results, is a bar chart which summarises the number of deviations from literature within given error ranges. This bar chart is reported in Appendix X, \jnote{FIG REF}.

Comparing W1BD results to literature, the MAE is 0.82 \kcalmol, with the majority of the data falls within 1--2 \kcalmol of each other. This suggests that both W1BD is consistent with the literature values. There are, however, two large outliers: dimethylsulfoxide\footnotemark~ and \emph{N,N}-dimethylacetamide, with experiment underestimating the BDEs by -8.03 and -8.22 \kcalmol, respectively. This result is consistent amongst all composite method, verifying the inaccuracy of these results.

\footnotetext{The experimental BDE for dimethylsulfoxide was previously identified as accurate by Salamone et al.\cite{Salamone2012}}

The best agreement with both W1BD and literature is the ROCBS-QB3 method (MAE = 0.18 (1.64) \kcalmol). In comparison, CBS-QB3 has an MAE = 0.32 (1.88) \kcalmol, while CBS-APNO has an MAE = 0.20 (1.40) \kcalmol.  The G4 method deviates from the W1BD reference by about 0.5 \kcalmol more, however, it appears to give reasonable agreement with experimental results (MAE = 0.70 (1.21) mol). The use of the MP2 variant of G4 gives somewhat questionable results, with an MAE of 0.88 (1.60) \kcalmol, as well as a large outlier of 6.23 kcal/mol that is not present in the other data from composite methods.

An alternative method for visualising these data is through the use of one-to-one plots, in which BDEs from two methods are directly compared. An ideal plot should have a slope = 1 and y-intercept = 0. These plots are reported in Appendix X, \jnote{FIG REF}, where it can once seen that ROCBS-QB3 performs best for the calculation of BDEs while G4(MP2) performs worst. Given these data, and considering the relative computational cost, the ROCBS-QB3 method is recommended for the efficient calculation of accurate BDEs, particularly for large molecules for which more expensive computational methods are not possible. Importantly, we can now confidently continue investigating the BEP relationships using reliable calculated BDE data from the ROCBS-QB3 method.

\section{Analysis of the Bell-Evans-Polanyi Relation}



\newpage
\jnote{Data to be moved to appendix}
\begin{figure}[htb]
  \centering
  \includegraphics[width=\textwidth]{figures/bde-mae-barchart}
  \caption{Summary of deviations of BDEs from reference for composite quantum chemical methods. Errors are relative to Reference~\citenum{Luo2002}. Numbers out of 49 represent the total number of data points which were computed for the given method.}
\end{figure}

\begin{figure}
  \centering
  \includegraphics[width=0.7\textwidth]{figures/lit-w1bd}
\end{figure}

\begin{figure}
\centering
\begin{minipage}{8cm}
  \centering
  \includegraphics[width=\textwidth]{figures/w1bd-cbsqb3}
\end{minipage}%
\begin{minipage}{8cm}
  \centering
  \includegraphics[width=\textwidth]{figures/lit-cbsqb3}
\end{minipage}
\end{figure}

\begin{figure}
\centering
\begin{minipage}{8cm}
  \centering
  \includegraphics[width=\textwidth]{figures/w1bd-rocbsqb3}
\end{minipage}%
\begin{minipage}{8cm}
  \centering
  \includegraphics[width=\textwidth]{figures/lit-rocbsqb3}
\end{minipage}
\end{figure}

\begin{figure}
\centering
\begin{minipage}{8cm}
  \centering
  \includegraphics[width=\textwidth]{figures/w1bd-cbsapno}
\end{minipage}%
\begin{minipage}{8cm}
  \centering
  \includegraphics[width=\textwidth]{figures/lit-cbsapno}
\end{minipage}
\end{figure}

\begin{figure}
\centering
\begin{minipage}{8cm}
  \centering
  \includegraphics[width=\textwidth]{figures/w1bd-g4}
\end{minipage}%
\begin{minipage}{8cm}
  \centering
  \includegraphics[width=\textwidth]{figures/lit-g4}
\end{minipage}
\end{figure}

\begin{figure}
\centering
\begin{minipage}{8cm}
  \centering
  \includegraphics[width=\textwidth]{figures/w1bd-g4mp2}
\end{minipage}%
\begin{minipage}{8cm}
  \centering
  \includegraphics[width=\textwidth]{figures/lit-g4mp2}
\end{minipage}
\end{figure}

\begin{landscape}
%\singlespacing
\newcommand{\tabBDE}[2][0.7]{\includegraphics[scale=#1]{figures/#2.eps}}
\setlength\LTleft{-1.5cm}
\begin{longtable}{m{3.5cm} >{\centering}m{3.5cm} | >{\centering}m{0.8cm} >{\centering}m{0.9cm} >{\centering}m{3cm} >{\centering}m{0.9cm} m{0em}}
\caption[Bond dissociation enthalpies of the species used to investigate the accuracy of composite methods.]{Bond dissociation enthalpies of the species used to investigate the accuracy of composite methods. Structures show an explicit C-H bond for that which is cleaved. All values are in \kcalmol. Statistics are listed at the bottom of the table.} \label{tab:bde-calc} \\
Molecule                         & Structure &  Lit.     &   W1BD   &   ROCBS-QB3 &    G4   &\\
\hline
% \endfirsthead
% Molecule                         & Structure &  Lit.     &   W1BD   &   ROCBS-QB3 &    G4   &\\
% \hline
% \endhead
1,3-pentadiene                   & \tabBDE{BDEs/13pentadiene} &  83.0     &   82.9   &     81.7    &   81.6  &\\
1,4-cyclohexadiene               & \tabBDE{BDEs/14cyclohexadiene} &  76.0     &   76.3   &     75.0    &   75.2  &\\
1,4-diazabicyclo[2.2.2]-octane   & \tabBDE{BDEs/DABCO} &  93.4     &          &     98.8    &   96.7  &\\
1,4-pentadiene                   & \tabBDE{BDEs/14pentadiene} &  76.6     &   76.2   &     75.0    &   75.1  &\\
2,2-dimethylbutane               & \tabBDE{BDEs/22dimethylbutane} &  98.0     &   99.3   &     99.3    &   97.5  &\\
2,3-dimethylbutane               & \tabBDE{BDEs/23dimethylbutane} &  95.4     &   97.8   &     97.8    &   96.2  &\\
2-methylbutane                   & \tabBDE{BDEs/2methylbutane} &  95.8     &   97.3   &     97.1    &   95.9  &\\
9,10-dihydroanthracene           & \tabBDE[0.5]{BDEs/dhanthracene} & 76.3     &          &     78.1    &          &\\
Acetaldehyde                     & \tabBDE{BDEs/acetaldehyde} &  94.3     &   95.9   &     95.7    &   94.9  &\\
Acetone                          & \tabBDE{BDEs/acetone} &  96.0     &   96.9   &     96.7    &   95.4  &\\
Acetonitrile                     & \tabBDE{BDEs/acetonitrile} &  97.0     &   96.9   &     96.6    &   96.3  &\\
Adamantane (2$^\circ$)           & \tabBDE{BDEs/adm-sec} &  98.4     &          &    100.4    &   97.8  &\\
Adamantane (3$^\circ$)           & \tabBDE{BDEs/adm-tert} &  96.2     &          &     99.9    &         &\\
Benzaldehyde                     & \tabBDE{BDEs/benzaldehyde} &  88.7     &          &     91.4    &   89.3  &\\
Benzene                          & \tabBDE{BDEs/benzene} & 112.9     &  113.1   &    113.0    &         &\\
Benzyl Alcohol                   & \tabBDE{BDEs/benzylalcohol} &  79.0     &          &     83.2    &   83.4  &\\
Cumene                           & \tabBDE{BDEs/cumene} &  83.2     &          &     86.9    &   86.9  &\\
Cycloheptane                     & \tabBDE{BDEs/cycloheptane} &  94.0     &          &     95.8    &   93.9  &\\
Cyclohexane                      & \tabBDE{BDEs/cyclohexane} &  99.5     &   99.2   &     99.3    &   97.5  &\\
Cyclooctane                      & \tabBDE{BDEs/cyclooctane} &  94.4     &          &     92.4    &   90.2  &\\
Cyclopentane                     & \tabBDE{BDEs/cyclopentane} &  95.6     &   96.3   &     96.3    &   95.6  &\\
Cyclopropane                     & \tabBDE{BDEs/cyclopropane} & 106.3     &  109.0   &    109.2    &  108.2  &\\
Dibenzyl ether                   & \tabBDE[0.4]{BDEs/dibenzylether} &  85.8     &          &     82.7    &         &\\
Diethyl ether                    & \tabBDE{BDEs/diethylether} &  93.0     &   95.3   &     95.5    &   93.8  &\\
Dimethylamine                    & \tabBDE{BDEs/dimethylamine} &  94.2     &   92.6   &     92.8    &   92.0  &\\
Dimethylsulfoxide                & \tabBDE{BDEs/DMSO} &  94.0     &  102.0   &    102.3    &  100.9  &\\
Dioxane                          & \tabBDE{BDEs/dioxane} &  96.5     &   97.3   &     97.6    &   95.7  &\\
Diphenylmethane                  & \tabBDE[0.5]{BDEs/diphenylmethane} &  84.5     &          &     82.8    &         &\\
Ethane                           & \tabBDE{BDEs/ethane} & 100.5     &  101.3   &    101.5    &  100.7  &\\
Ethylbenzene                     & \tabBDE{BDEs/ethylbenzene} &  85.4     &          &     87.6    &   87.6  &\\
Ethylene                         & \tabBDE{BDEs/ethylene} & 110.9     &  110.8   &    110.9    &  109.9  &\\
Fluorene                         & \tabBDE[0.5]{BDEs/fluorene} &  82.0     &          &     81.9    &         &\\
Formaldehyde                     & \tabBDE{BDEs/formaldehyde} &  88.0     &   88.6   &     88.9    &   88.2  &\\
Hexamethyl-phosphoramide         & \tabBDE{BDEs/HMPA} &           &          &     93.9    &         &\\
Indene                           & \tabBDE{BDEs/indene} &  83.0     &          &     80.1    &   79.0  &\\
Methane                          & \tabBDE{BDEs/methane} & 105.0     &  105.0   &    105.2    &  104.5  &\\
Methanol                         & \tabBDE{BDEs/methanol} &  96.1     &   96.4   &     96.8    &   96.0  &\\
Methylamine                      & \tabBDE{BDEs/methylamine} &  93.9     &   93.1   &     93.3    &   92.7  &\\
Morpholine                       & \tabBDE{BDEs/morpholine} &  92.0     &          &     93.3    &   91.8  &\\
N,N-dimethylacetamide            & \tabBDE{BDEs/DMA} &  91.4     &   99.6   &     99.5    &   97.6  &\\
Piperazine                       & \tabBDE{BDEs/piperazine} &  93.0     &   93.4   &     93.5    &   91.9  &\\
Piperidine                       & \tabBDE{BDEs/piperidine} &  89.5     &   92.1   &     92.2    &   90.7  &\\
Propane                          & \tabBDE{BDEs/propane} & 100.9     &  101.6   &    101.8    &  100.7  &\\
Pyrrolidine                      & \tabBDE{BDEs/pyrrolidine} &  89.0     &   90.8   &     90.7    &   89.5  &\\
Tetrahydro-2H-pyran              & \tabBDE{BDEs/oxane} &  96.0     &   96.3   &     96.5    &   94.7  &\\
Tetrahydrofuran                  & \tabBDE{BDEs/THF} &  92.1     &   93.7   &     93.8    &   92.2  &\\
Toluene                          & \tabBDE{BDEs/toluene} &  89.7     &   90.5   &     89.7    &   89.8  &\\
Trichloromethane                 & \tabBDE{BDEs/trichloromethane} &  93.8     &   93.5   &     93.7    &   92.4  &\\
Triethylamine                    & \tabBDE{BDEs/triethylamine} &  90.7     &          &     91.2    &   89.4  &\\
Trifluormethane                  & \tabBDE{BDEs/trifluoromethane} & 106.4     &  107.2   &    107.4    &  105.8  &\\
\hline
\textbf{Statistics}              & & Lit.      &  W1BD    &  ROCBS-QB3 &     G4   &\\
\hline
Number of BDEs               &($N$) &    49     &     33   &      50    &     43   &\\
MAE (Lit.)                       & &           &   0.82   &    1.64    &   1.21   &\\
Max. Error                       & &           &   1.59   &    3.15    &   4.19   &\\
Min. Error                       & &           &  -8.22   &   -8.25    &  -6.86   &\\
MAE (W1BD)                       & &           &          &    0.18    &   0.70   &\\
Max. Error                       & &           &          &    1.26    &   2.05   &\\
Min. Error                       & &           &          &   -0.35    &   0.37   &\\
\end{longtable}
\setlength\LTleft{0pt}
\setlength\LTright{0pt}

\end{landscape}

\begin{longtable}{m{3.1cm} | c c c}
\caption{Summary of experimental rate constants and literature\cite{Luo2002} bond dissociation enthalpies (BDEs).} \label{tab:expt-bde} \\
\centering
 Molecule                       & $k_H$ \Ms          & Normalized $k_H$ \Ms & BDE \kcalmol \\
\toprule
 1,4-cyclohexadiene             & $ 6.60 \times 10^7$ & $1.65 \times 10^7 $ &        76 \\
 1,4-diazabicyclo-[2.2.2]octane  & $ 9.60 \times 10^6$ & $8.00 \times 10^5 $ &      93.4 \\
 2,2-dimethylbutane             & $ 9.50 \times 10^4$ & $4.75 \times 10^4 $ &        98 \\
 2,3-dimethylbutane             & $ 5.60 \times 10^5$ & $2.80 \times 10^5 $ &      95.4 \\
 9,10-dihydroanthracene         & $ 5.04 \times 10^7$ & $1.26 \times 10^7 $ &      76.3 \\
 Acetone                        & $ < 1 \times 10^4 $ & $2 \times 10^3    $ &        96 \\
 Acetonitrile                   & $ < 1 \times 10^4 $ & $2 \times 10^3    $ &        97 \\
 Adamantane (2$^\circ$)                & $ 6.90 \times 10^6$ & $5.75 \times 10^5 $ &      98.4 \\
 Adamantane (3$^\circ$)                & $ 6.90 \times 10^6$ & $1.73 \times 10^6 $ &      96.2 \\
 Benzaldehyde                   & $ 1.20 \times 10^7$ & $1.20 \times 10^7 $ &      88.7 \\
 Benzyl Alcohol                 & $ 2.97 \times 10^6$ & $1.49 \times 10^6 $ &        79 \\
 Cumene                         & $ 5.60 \times 10^5$ & $5.60 \times 10^5 $ &      83.2 \\
 Cycloheptane                   & $ 2.20 \times 10^6$ & $1.57 \times 10^5 $ &        94 \\
 Cyclohexane                    & $ 1.10 \times 10^6$ & $9.17 \times 10^4 $ &      99.5 \\
 Cyclooctane                    & $ 2.98 \times 10^6$ & $1.86 \times 10^5 $ &      94.4 \\
 Cyclopentane                   & $ 9.54 \times 10^6$ & $9.54 \times 10^5 $ &      95.6 \\
 Dibenzyl ether                 & $ 5.60 \times 10^6$ & $1.40 \times 10^6 $ &      85.8 \\
 Diethyl ether                  & $ 2.60 \times 10^6$ & $6.50 \times 10^5 $ &        93 \\
 Dimethylsulfoxide              & $ 1.80 \times 10^4$ & $6.00 \times 10^3 $ &        94 \\
 Dioxane                        & $ 8.20 \times 10^5$ & $1.03 \times 10^5 $ &      96.5 \\
 Diphenylmethane                & $ 8.71 \times 10^5$ & $4.36 \times 10^5 $ &      84.5 \\
 Ethylbenzene                   & $ 7.90 \times 10^5$ & $3.95 \times 10^5 $ &      85.4 \\
 Hexamethyl-phorsphoramide       & $ 1.87 \times 10^7$ & $1.04 \times 10^6 $ &           \\
 Morpholine                     & $ 5.00 \times 10^7$ & $1.25 \times 10^7 $ &        92 \\
 Piperazine                     & $ 2.4 \times 10^8 $ &                &        93 \\
 Piperidine                     & $ 1.2 \times 10^8 $ &                &      89.5 \\
 Pyrrolidine                    & $ 1.1 \times 10^8 $ &                &        89 \\
 Tetrahydro-2H-pyran            & $ 1.4 \times 10^6 $ &                &        96 \\
 Tetrahydrofuran                & $ 5.8 \times 10^6 $ &                &      92.1 \\
 Toluene                        & $ 1.85 \times 10^5$ & $6.17 \times 10^4 $ &      89.7 \\
 Triethylamine                  & $ 2.10 \times 10^8$ & $3.5 \times 10^7  $ &      90.7 \\
 Triphenylmethane               & $ 3.04 \times 10^5$ & $3.04 \times 10^5 $ &        81 \\
\end{longtable}


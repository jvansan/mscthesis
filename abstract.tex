% !TEX root = diss.tex

% Abstract
\begin{abstract}                % Mandatory -  maximum 350 words

\begin{doublespace}

Hydrogen atom transfer (HAT) reactions are a fundamental step in many biological
processes, but can initiate the free-radical induced oxidation of biomaterials.
Although HAT reactions appear fundamentally elementary, there are many poorly
understood factors that influence HAT. In this thesis, three aspects of HAT
reactivity are investigated using quantum chemical technique.

First, the importance of pre-reaction complex formation in considering the
kinetics of HAT reactions was investigated. For a set of nearly-thermoneutral
HAT reactions involving oxygen-centred radicals, the relationship between
pre-reaction complex non-covalent binding energies and Arrhenius pre-exponential
factors (A-factors) is investigated. It is demonstrated that for HAT reactions
that take place through similar mechanisms, there is a strong correlation
between pre-reaction complex binding energies and A-factors. This suggests that
non-covalent interactions may directly affect the kinetics of certain HAT
reactions.

Next, the relationship between bond dissociation energies (BDEs) and reaction
rates for abstraction of a hydrogen from a \ch{C-H} bond by the \cumo\ radical
are investigated in the context of the Bell-Evans-Polanyi (BEP) principle. The
applicability of the BEP principle is examined by exploring a hypothesis: If the
BEP principle is a valid linear free-energy relationship, there should exist two
linear relationships for BDE against the logarithm of HAT rate constant, one for
incipient radicals that are allylic or benzylic, and one for alkyl radical. It
is demonstrated that there is a reasonably strong correlation for
allylic/benzylic \ch{C-H} bonds, however not for alkyl ones. The BEP principle
should not be used for quantitative prediction, however it remains useful as a
conceptual framework.

Finally, the effect of non-redox active metal cations on HAT reactions involving
small models for proteins and oxygen-centred radicals is studied. Previous
experimental evidence demonstrated that Lewis acid-base interactions between
metal cations and substrates can inhibit HAT reactions, and that the cations may
serve as a form of chemo-protection in biological systems. The results herein
demonstrate that metal-substrate interactions can deactivate certain \ch{C—H}
bonds, however metal-radical interactions may promote HAT reactions. On the
basis of these limited results, non-redox active metal cations may not act as
natural chemo-protective agents.

\end{doublespace}

\end{abstract}

% !TEX root = diss.tex

\chapter{Do non-redox active metal cations have the potentials to behave as chemo-protective agents? The Effects on Metal Cations on HAT Reaction Barrier Heights}
\label{ch:hat}

\section{Introduction}

Metal cations are ubiquitous in biological systems and play an important role in biological function. As such, there is a great deal of interest in studying metals in biological systems. Proteins in particular are often associated with metals, and in the worldwide Protein Data Bank,\cite{Harding2010, Berman2007} over one-third of crystal structures contain metals. Redox active metals, such as copper and iron, act as co-factors in metalloenzymes for important catalytic processes.\cite{Atkins2010}

Non-redox active metal cations are equally as important in biological function as redox active metals, where they are essential to protein structure and function, along with cellular and neuronal signaling.\cite{Karp1999} Sodium and calcium ions are most abundant extracellurly, while potassium and magnesium are dominant inside of cells. While specific ionic concentrations vary dramatically dependening on physiological conditions, estimates for equilibrium concentrations in both mammalian heart cells\cite{Ingwall2006} and blood plasma\cite{daSilva2001} are listed in~\ref{tab:metalconc}. As sodium and magnesium are most abundant alkali and alkaline earth metals found in biologically relevant systems, they are of prime interest for investigation.

\begin{table}[!htbp]
  \caption{Ionic concentrations inside a mammalian heart cell and in the blood plasma. Concentrations are in units of mM. Values are rounded to one significant figure. Data are from Ref. \protect\citenum{Ingwall2006} and \protect\citenum{daSilva2001}.}
  \label{tab:metalconc}
\begin{tabular}{l c c}
  Ion Conc. & Mammalian Cells & Blood Plasma \\
  \hline
  \ch{Na^+} & 10 & 100--200 \\
  \ch{Mg^{2+}} & 10 & 1 \\
  \ch{K^+} & 100 & 4 \\
  \ch{Ca^{2+}} & 0.1 & 2
\end{tabular}
\end{table}

Extensive crystallographic surveys indicate that metals bind predominantly to oxygen centres in proteins.\cite{Harding1999, Harding2004, Hsin2008} Divalent metals are most often found bound directly to proteins. Calcium binds anywhere from 4 to 6 binding sites in protein crystal structures, while magnesium binds only 1 or 2. Monovalent metals, on the other hand, are often heavily solvated and so they appear in solvent cavities of proteins, although sodium or potassium are sometimes found bound directly to carbonyl or carboxylate oxygen centres.\cite{Harding2010}

A great deal of research has focussed on \ch{Ca^{2+}} in the context of reactive oxygen-centred radical production.\cite{Goerlach2015} Specifically, \ch{Ca^{2+}} ions are important in the mitochondria, where, depending on physiological conditions and concentrations, they can act as inhibitor or promoters of free-radical production in the electron transport chain.\cite{AdamVizi2010} In explanation is that \ch{Ca^{2+}} induce conformational changes of the proteins involved in the electron transport chain which are responsible for radical generation.\cite{Brookes2004} Mitochondrial free-radicals, when present in moderate amounts, can act as cell signalling molecules to activate pro-growth responses.\cite{Sullivan2014} However, ``dysfunctional'' mitochondria can produce excess radicals leading to oxidative damage which has been linked to degenerative diseases.

Given the significant importance alkali and alkaline earth metals play in biological systems, their impact on protein oxidation must be considered. However, until recently, kinetic studies of protein oxidation have not investigated the mechanistic role of non-redox active metals. In a series of three papers,\cite{Salamone2013a, Salamone2015metals, Salamone2016} Bietti and colleagues have shown that alkali and alkaline earth metals have an inhibitory effect on HAT reactions involving \cumo\ and organic substrates. Some of the experimental rate constants from these papers are summarized in~\ref{tab:hat-metals}. All rate constants were obtained by time-resolved LFP in nitrogen or argon-saturated acetonitrile (MeCN) at 298 K, as was previously described in Section~\ref{sec:hat-methods}. All of the experimental results have been rationalized on the basis of Lewis acid metals cations interactions with Lewis basic substrates.

\bgroup
\def\arraystretch{1.2}%  1 is the default, change whatever you need
\begin{table}
  \caption{Table summary of experimental data - Needs completion}
  \label{tab:hat-metals}
  \hspace*{-1.2cm}
  \begin{tabular}{l l c c}
    Substrate & Conditions & $k_H$ (\Ms) & $k_H$(MeCN)/$k_H$(M$^{n+}$) \\
    \hline
    1,4-cyclohexadiene &    & 6.7\E{7} & \\
    (CHD)  & \ch{LiClO4} 1.0 M & 7.5\E{7} & 0.89 \\
     & \ch{Mg(ClO4)2} 1.0 M & 7.0\E{7} & 0.96 \\
    tetrahydrofuran &   & 5.7\E{6} & \\
    (THF) & \ch{LiClO4} 1.0 M & 2.9\E{6} & 1.7 \\
     & \ch{LiOTf} 1.0 M & 2.8\E{6} & 2.0 \\
     & \ch{Mg(ClO4)2} 1.0 M & 1.8\E{6} & 3.2 \\
    triethylamine &  & 2.0\E{8} & \\
    (TEA) & \ch{LiClO4} 1.0 M & 9.4\E{7} & 2.1 \\
     & \ch{Mg(ClO4)2} 0.005 M & $<$1\E{6} & $>$200 \\
    $N,N$-dimethylformamide & & 1.2\E{6} & \\
    (DMF) & \ch{LiClO4} 0.5 M & $k_{H1}$ = 8.9\E{5} & 1.3 \\
      & & $k_{H2}$ = 1.5\E{6} & 0.80 \\
      & \ch{NaClO4} 0.2 M & $k_{H1}$ = 9.6\E{5} & 1.3 \\
      & & $k_{H2}$ = 1.4\E{6} & 0.86 \\
      & \ch{Mg(ClO4)2} 0.2 M & $k_{H1}$ = 5.8\E{5} & 2.1 \\
      & & $k_{H2}$ = 1.1\E{6} & 1.1 \\
      & \ch{Ca(ClO4)2} 0.2 M & $k_{H1}$ = 1.0\E{6} & 0.83 \\
    $N,N$-dimethylacetamide &  & 1.2\E{6} & \\
    (DMF) & \ch{LiClO4} 0.2 M & $k_{H1}$ = 8.5\E{5} & 1.4 \\
      & & $k_{H2}$ = 1.5\E{6} & 0.8 \\
      & \ch{NaClO4} 0.2 M & $k_{H1}$ = 1.1\E{6} & 1.1 \\
      & & $k_{H2}$ = 1.3\E{6} & 0.92 \\
      & \ch{Mg(ClO4)2} 0.2 M & $k_{H1}$ = 4.7\E{5} & 2.6 \\
      & & $k_{H2}$ = 2.4\E{5} & 5.0 \\
      & & $k_{H3}$ = 1.1\E{6} & 1.1 \\
      & \ch{Ca(ClO4)2} 0.2 M & $k_{H1}$ = 1.2\E{6} & 1.0
  \end{tabular}
\end{table}
\egroup

Firstly, for hydrocarbons, cyclic ethers, and tertiary amines, \cumo\ hydrogen abstraction rate constants in the presence of excess concentrations of lithium and magnesium salts were measured.\cite{Salamone2013a} In the presence of \ch{LiClO4} and \ch{Mg(ClO4)2}, the rate of abstraction by \cumo\ from 1,4-cyclohexadiene (CHD) increases very slightly. Since CHD has no Lewis basic centres, the increase in HAT rate constant was explained on the basis of metal cation interactions with \cumo, very slightly increasing the hydrogen abstraction ability by withdrawing electron density from the aromatic ring. Metal cations were also shown to increase the unimolecular decay of \cumo\ by $\beta$-scission (See Section~\ref{sec:hat-methods}). The largest kinetic effect was observed with \ch{LiClO4} with $k_\beta$ = 1.8\E{6} $s^{-1}$, which is a
roughly 3-fold increase as compared to the rate in MeCN at 298 K ($k_\beta$\cite{Avila1995} = 6.3\E{5} $s^{-1}$). This effect is significantly less than the observed kinetic solvent effect on \cumo\ $\beta$-scission measured in \ch{H2O} or 2,2,2-trifluoroethanol ($k_\beta$ = 1.0\E{7} and 6.1\E{6} $s^{-1}$, respectively).\cite{Bietti2005, Neta1984} Therefore, the kinetic effects of these alkali and alkaline metal salts interacting via Lewis acid-base interactions with the oxygen-centre of \cumo\ are less than the effects of hydrogen-bonding by solvents.

Next, the HAT rate constants for abstraction from tetrahydrofuran (THF) decrease in the presence of non-redox active metal salts. Both \ch{LiClO4} and \ch{LiOTf} decrease $k_H$ by a factor of about 2, indicating the nature of the counter-anion plays a negligible role in the Lewis acid-base interactions between metal cations and substrates. The addition of \ch{Mg(ClO4)2} has a greater effect on HAT reactivity, decreasing $k_H$ by a factor of 3. Magnesium ions are a stronger Lewis acid than lithium,\cite{Fukuzumi2002} supporting the notion of Lewis acid-base interactions between the oxygen lone-pair and the metal cations. The decrease in $k_H$ has been partially attributed to the reduction in hyperconjugative overlap between the oxygen lone-pair and the neighbouring \ch{C-H} $\sigma^*$ anti-bonding orbital (See~\ref{fig:THF}), as a consequence of the metal cation withdrawing electron density from the oxygen lone-pair.

A 2-fold decrease in $k_H$ for the tertiary amine, triethylamine (TEA), is observed upon the addition of \ch{LiClO4}, for which an analogous orbital interaction explanation is also appropriate. Interestingly, the addition of 1.0 M \ch{Mg(ClO4)2} was reported to immediately form a precipitate to form. This precipitate was identified as the formation of a strong TEA-\ch{Mg^{2+}} Lewis acid-base adduct. This observation is once again consistent with the stronger Lewis acidity of \ch{Mg^{2+}} as compared to \ch{Li^+}, and also the significantly greater Lewis basicity of TEA vs THF.\cite{Salamone2013a, Reichardt2010} It was also pointed out that MeCN will competitively bind with metal cations, however it is a weaker Lewis base than both THF and TEA. Measurements of $k_H$ for HAT between \cumo\ and TEA in the presence of 0.005 M \ch{Mg(ClO4)2} were successful only up until [TEA] = 9.6 mM, at which point a precipitate began to form. Nonetheless, an upper limit to the hydrogen abstraction rate constant was estimated as $k_H <$ 1\E{6} \Ms, or at least a 200 fold decrease relative to no metal salt. Very similar results for bulkier tertiary amines were also obtained. Thus, the presence of strong Lewis acids in the presence of Lewis basic sites on hydrogen atom donors can deactivate \ch{C-H} bonds.

Next, we turn to the more relevant models for the work of this thesis, the tertiary amides $N,N$-dimethylformamide (DMF) and $N,N$-dimethylacetamide (DMA). As with THF, normal hyperconjugative overlap between the conjugated amide $\pi$-system and the adjacent \ch{C-H} $\sigma^*$ anti-bonding orbitals weakens the C-H bonds. Therefore, metal binding to the amide oxygen-centre should result in a decrease in this orbital interaction, strengthen the C-H bonds, and decrease HAT reactivity. In their study, \citet{Salamone2015metals} measured \cumo\ abstraction rate constants from DMF and DMA in the presence of stoichiometric equivalents of \ch{LiClO4}, \ch{LiOTf}, \ch{NaClO4}, \ch{Mg(ClO4)2}, and \ch{Ca(ClO4)2} (in contrast to the excess used in Reference~\citenum{Salamone2013a}). Figure~\ref{fig:k-metals-mg}a,b shows the plots of $k_{obs}$ against [substrate] for the reactions of \cumo with DMF and DMA in MeCN containing 0.2 M \ch{Mg(ClO4)2}, respectively. For both DMF and DMA, there are three distinct regions in the plots: weak C-H bond activation for [amide]/[\ch{Mg^{2+}}]$\leq 2$, followed by strong C-H bond deactivation for 2$<$[amide]/[\ch{Mg^{2+}}]$\leq$4, and no deactivation for [amide]/[\ch{Mg^{2+}}]$<$4.

\begin{figure}[!htbp]
  \includegraphics[width=\textwidth]{figures/kH-dma-dmf-mgclo42.png}
  \caption[Plot of observed rate constant against concentration of DMF and DMA for reaction with \cumo\ at 298 K in the presence of 0.2 M \ch{Mg(ClO4)2}.]
  {\textbf{a)} Plot of observed rate constant against concentration of DMF for reaction with \cumo\ at 298 K in the presence of 0.2 M \ch{Mg(ClO4)2}. 0--0.4 M [DMF] range (black circles), $k_{H1}$ = 5.8\E{5} \Ms; 0.8--2.2 M [DMF] range (white circles), $k_{H2}$ = 1.3\E{6} \Ms.
  \textbf{b)} Plot of observed rate constant against concentration of DMA for reaction with \cumo\ at 298 K in the presence of 0.2 M \ch{Mg(ClO4)2}. 0--0.4 M [DMA] range (black circles), $k_{H1}$ = 4.7\E{5} \Ms; 0.4--0.8 M [DMA] range (grey circles), $k_{H2}$ = 2.4\E{5} \Ms; 0.8--2.2 M [DMA] range (white circles), $k_{H3}$ = 1.1\E{6} \Ms. Reprinted with permission from Reference~\protect\citenum{Salamone2015metals}. Copyright 2015 American Chemical Society.}
  \label{fig:k-metals-mg}
\end{figure}

The addition of both \ch{LiClO4} and \ch{LiOTf} decrease to a similar extent the rate constants for abstraction from DMF and DMA by \cumo. However, in contrast to \ch{Mg(ClO4)2}, the lithium salts strongly deactivate \ch{C-H} bonds for 2 equivalents, followed by weak deactivation for another 2 equivalents, and no deactivation for [amide]/[\ch{Li^{+}}]$<$4. Salamone et al. were not able to give a clear cut explanation, but suggest that the different patterns are a result of differences in charge density, which is greater for \ch{Mg^{2+}} than \ch{Li^+}, as well as different coordination geometries of the two ions. A coordination number of 4 is most common for \ch{Li^+}, while an octahedral geometry with the coordination of 6 ligands is almost always observed for \ch{Mg^{2+}}.\cite{Babu2013, Dudev2014} As a result, interactions of the ions with solvent and counter-anions are suggested to be more important for \ch{Mg^{2+}} than \ch{Li^+}.

\ch{NaClO4} and \ch{Ca(ClO4)2} influence HAT between \cumo\ and DMA to different extents than both \ch{LiClO4} and \ch{Mg(ClO4)2}. Figure~\ref{fig:k-metals-naca}a,b shows the plots of $k_{obs}$ against [substrate] for the reactions of \cumo with DMA in MeCN containing 0.2 M \ch{NaClO4} and \ch{Mg(ClO4)2}, respectively. For \ch{NaClO4}, an almost negligible deactivation of \ch{C-H} bonds is observed for up to 4 equivalents of DMA. This was explained on the basis of the weaker Lewis acidity of \ch{Na^+} as compared to \ch{Li^+}. With regards to \ch{Ca(ClO4)2}, binding to DMA fully deactivates \ch{C-H} bond abstraction up to 4 equivalents of DMA. The first region of~\ref{fig:k-metals-naca}b ([DMA] = 0--0.2 M, black circles) represents the decrease in $k_\beta$ of \cumo as \ch{Ca^{2+}} preferentially binds to DMA over \cumo. Interstingly, for both DMF and DMA, the same experiments in dimethyl sulfoxide (DMSO) solvent show no inhibition of HAT reactivity by metal cations. This was rationalized on the basis of the stronger Lewis basicity of DMSO as compared to both MeCN and the amides, thus the metals preferentially bind solvent rather than amide substrate.

\begin{figure}[!htbp]
  \includegraphics[width=\textwidth]{figures/exptdma-na-ca.png}
  \caption[Plot of observed rate constant against concentration of DMA for reaction with \cumo\ at 298 K in the presence of 0.2 M \ch{NaClO4} and \ch{Mg(ClO4)2}.]
  {\textbf{a)} Plot of observed rate constant against concentration of DMA for reaction with \cumo\ at 298 K in the presence of 0.2 M \ch{NaClO4}. 0--0.8 M [DMA] range (black circles), $k_{H1}$ = 9.6\E{5} \Ms; 0.8--1.4 M [DMA] range (white circles), $k_{H2}$ = 1.4\E{6} \Ms.
  \textbf{b)} Plot of observed rate constant against concentration of DMA for reaction with \cumo\ at 298 K in the presence of 0.2 M \ch{Ca(ClO4)2}. 0.8--1.7 M [DMA] range (white circles), $k_{H1}$ = 1.2\E{6} \Ms. Adapted with permission from Reference~\protect\citenum{Salamone2015metals}. Copyright 2015 American Chemical Society.}
  \label{fig:k-metals-naca}
\end{figure}

Finally, \citet{Salamone2016} examined the effects of substrate structure on HAT reaction between \cumo and tertiary alkanamides in the presence of alkali and alkaline earth metal ions. For $N,N$-dialkylacetamides, the steric bulk of the $N$-alkyl groups was previously characterized.\cite{Salamone2014} Steric repulsion between \cumo\ and the $N$-alkyl groups can decreases the HAT rate constant, as evident by the 3-fold decrease in $k_H$ in going from DMA to $N,N$-diisobutylacetamide (DIA; 1.2\E{6} and 3.1\E{5} \Ms, respectively). For reactions of \cumo\ with DIA addition of 0.2 M \ch{LiClO4} or \ch{Ca(ClO4)2} to results in the same trends in \ch{C-H} bond deactivation observed for DMA. This indicates that the influence of metal cation-substrate binding is not significantly influences by the steric bulk of $N$-alkyl groups.  The same is true for the addition of 0.2 M \ch{Mg(ClO4)2} to abstraction from DIA by \cumo, as shown in~\ref{fig:k-dia-mg}. Once again, an slight decrease in reactivity is observed for the first 2 equivalents of DIA, followed by strong C-H bond deactivation for an additional two equivalents, and no deactivation beyond that. No additional insight was provided by Salamone et al. as to the reason for this reactivity. The plausible explanation provided was once again that \ch{Mg^{2+}} has a high charge density. These results show that Lewis acid-base interactions between alkali or alkaline earth metal cations can greatly depress hydrogen abstraction by alkyoxyl radicals.

\begin{figure}[!htbp]
  \includegraphics[width=0.6\textwidth]{figures/exptdia-mg.png}
  \caption[Plot of observed rate constant against concentration of DIA for reaction with \cumo\ at 298 K in the presence of 0.2 M \ch{Mg(ClO4)2}.]
  {Plot of observed rate constant against concentration of DIA for reaction with \cumo\ at 298 K in the presence of 0.2 M \ch{Mg(ClO4)2}. 0--0.4 M [DIA] range, $k_{H1}$ = 3.6\E{5} \Ms; 0.8--1.4 M [DIA] range, $k_{H2}$ = 2.9\E{5} \Ms.
  Reprinted from Tetrahedron, 72, Salamone et al., Hydrogen atom transfer from tertiary alkanamides to the cumyloxyl radical. The role of substrate structure on alkali and alkaline earth metal ion induced \ch{C–H} bond deactivation, 7757--7763, 2016, with permission from Elsevier.}
  \label{fig:k-dia-mg}
\end{figure}

With these results in mind, I am interested in the possibility that alkali and alkaline earth metal cations found in biological system can protect \ch{C-H} bonds in proteins from HAT to reactive oxygen-centred radicals. However, the experimental results do not answer some of the key physico-chemical determinants which may make this possible. Specifically, I have composed several important research questions which remain unclear from the experimental results.

The first question which I have is one of methodology: Can DFT-based methods can accurately treat alkali/alkaline metal cation binding to organic substrates or radicals? There exists limited ab initio data describing these interactions.\cite{ Siu2001, Corral2003, Suarez2011, Baldauf2013} Therefore, I have conducted a benchmark quality study involving \ch{Li^+}, \ch{Na^+}, \ch{Mg^{2+}}, \ch{K^+}, and \ch{Ca^{2+}}. to the best of my knowledge, this represents the first systematic benchmark study of these metal cations with both organic substrates, and radicals.

Secondly, the nature of the binding of metal ions to substrates is still poorly described, especially given the odd stoichiometric effects observed for \ch{Mg(ClO4)2} with alkanamides. Specifically, I wish to address the range of these interactions, and how much the metals do effect the C-H being broken. Throughout this investigation I have utilized both \ch{Na^+} and \ch{Mg^{2+}} in my calculations. These metal ions were chosen to capture the large differences in Lewis acidity and ion size associated with these third-period ions, and because they are two of the most biologically relevant metal ions.

Thirdly, I address the effect that metal ions have on the HAT barrier heights.
Experiments demonstrate that under certain conditions the presence of metal ions can decrease HAT reactivity. If metal ions do effectively increase C-H bond strengths, this will certainly be a contributing factor to the free energy barrier. There will likely be additional factors such as polarization in the TS complex, or other effects of possible charge transfer from the substrate to metal ions. To investigate this, I have primarily studied HAT reactions involving DMA and oxygen-centred radicals. I was also interested in structural differences of the oxygen-centred radical, thus I have utilized the \bno and \cumo, which differ significantly in their ability to form strong pre-reaction complexes with hydrogen bond accepting substrates. Note that in this portion of the study, I was met with abundant technical difficulties related to optimizing TS structures including metal cations, and thus the scope of my results and discussion have been reduced.



I have also performed calculations with the bulkier DIA substrate to see if steric bulk does have an influence on the ability of a metal cation to effect HAT reactions.  These results

Finally, I wanted to address the effect of strong Lewis basicity in the hydrogen donating substrate. HAT reactions involving alkoxyl radicals and strong Lewis bases have been previously studied.%\cite{}



\section{Computational methods and details}

All quantum mechanical calculations were performed using either the Gaussian 09 software package,\cite{Frisch2009} or the TURBOMOLE software package.\cite{turbomole} Calculations for the benchmark quality data of metal binding to substrates were first optimized at the LC-$\omega$PBE-D3(BJ)/6-31+G(2d,2p) level of theory,\cite{Vydrov2006, Vydrov2006a, Grimme2010, Johnson2006} and later re-optimized with larger 6-311+G(3df,3pd) basis sets. Single-point energy calculations were then carried out using the coupled cluster methodology with single, double and perturbative triples with full core correlation, CCSD(T,Full), and various basis sets, as will be described in Section~\ref{sec:benchmark}. Final benchmark quality binding energies have been calculated using the F12$^*$ explicitly correlated method with Def2-QZVPPD primary basis sets and Def2-QZVPP auxiliary basis sets required for the resolution-of-the-identity (RI) approximation as implemented in TURBOMOLE. The RI approximation is used to reduce the computational cost associated with calculating MO integrals.\bibnote{For a detail description or the RI approximation and explicit correlation, see Ref.~\protect\citenum{OterodelaRoza2017}, Chapter 4.} At total of 31 different DFT-based methods with nearly complete 6-311+G(3df,3pd) and moderate 6-31+G(2d,2p) basis sets were tested both by single-point energy calculations on the benchmark structures. Geometry optimization calculations starting from the benchmark structures were also performed for three of the best performing DFT-based method, in order to verify their ability to capture the minimum energy bound structures.

To test the effects of metal cations on HAT barrier heights, calculations were first performed for the reactions not involving metal cations. Geometry optimizations were performed at the M05-2X\cite{Zhao2006}/6-31+G$^{**}$ level of theory. Transition state (TS) structures were obtained by first freezing the abstraction donor-hydrogen-acceptor bond lengths with multiple initial orientations. The frozen bonds were then relaxed to obtain the final TS structures, which were then used to identify the appropriate pre- and post-reaction complexes. All structures were subjected to harmonic vibrational frequency calculations, which were visualized using the Chemcraft program\cite{ccraft} to verify minima (or saddle-points with a single imaginary frequency connecting reactants to products for TS structures). Single-point energy calculations were performed at the M05-2X/6-311+G(2d,2p) level of theory. The effects of MeCN solvent were estimated by inclusion of the SMD\cite{Marenich2009} continuum solvent model in single-point energy calculations.

The inclusion of metal cations into the TS structures proved to be technically challenging. It was my expectation that I could simply include metal cations and necessary counter-anions into the minimum energy complex structures and re-optimize, however this was not the case. TS structures were once again obtained by constrained optimization with the inclusion of the metal cation and counter-anion and freezing the abstraction donor-hydrogen-acceptor bond lengths, providing a guess TS structure. However, in most cases the force constants (which are necessary for a TS optimization calculation) from the the guess TS structure were not representative of the true TS structure, thus force constants were recalculated for every step along the optimization, using the ``CalcAll'' keyword in Gaussian. Even using this method, many TS structures including metal cations failed to converge. The final TS structures were once again used to identify the appropriate pre- and post-reaction complexes.

Natural bond order (NBO) and natural population analysis (NPA) were utilized in order to investigate the electronic structures involved in the HAT reactions and the effects of metal cation binding.\cite{Reed1983, Reed1985, Glendening2012} Version 3.1 of the NBO software package,\cite{NBO3} as implemented in the Gaussian 09 package was used in all cases.\cite{Frisch2009}

\section{Benchmarking DFT based methods for the binding of alkali and alkaline earth metals to organic substrates and oxygen centred radicals}
\sectionmark{Benchmarking DFT based methods}
\label{sec:benchmark}

In order to confidently perform quantum mechanical mechanistic studies, the method of choice must be confidently calibrated. While DFT-based methods have been widely applied to these studied, few studies have previously investigated alkali and alkaline earth-metal cation binding to organic substrates.\cite{Corral2003, Suarez2011, Siu2001, Baldauf2013} Most importantly, benchmark quality data for a wide variety of metals binding to biologically relevant substrates and oxygen-centred radicals does not exist to calibrate DFT-based methods. Therefore, I proposed a benchmark study which incorporated all the biologically relevant alkali and alkaline earth-metal cations, models for dipeptides including amino acid side chains, oxygen-centred radicals, and solvents which are utilized in the experimental mechanistic studies involved in probing these systems. Unfortunately, due to computational restrictions (\emph{vide infra}), benchmark quality calculations on the originally proposed benchmark set were not possible. Full details of the originally proposed benchmark set are in Appendix~\ref{ap:hat},~\ref{fig:ap-set1}

Benchmark quality binding energies are generally calculated using the ``gold standard'' approach, CCSD(T)/CBS, where correlation consistent basis sets\cite{Marshall2011, Rezac2013} (cc-pV\emph{X}Z, \emph{X}=T,Q,5) developed by Dunning an\cite{Vydrov2006, Vydrov2006a}d co-workers are used for complete basis set extrapolation. These basis sets have limited availability for the metals of interest. Specifically, basis sets for K are not available, and only non-augmented basis sets for Li, Na, Mg, and Ca. It is necessary to include core-correlation of at least the first core shell in alkali and alkaline earth metals, thus it would be appropriate to use core valence basis sets such as cc-pCV$X$Z.\cite{Peterson2002} However, these basis sets are even more limited. Therefore, I originally chose the augmented version of the polarization consistent basis sets of Jensen and co-workers\cite{Jensen2001, Jensen2002, Jensen2002a, Jensen2003}  (aug-pc-\emph{N}, \emph{N}=2,3,4), which have been shown to converge to the CBS limit systematically\cite{Kupka2007} and are available for all the elements of interest.

While performing CCSD(T)/CBS calculations, I noticed that the metal cations (and neutral metal atoms), did not converge smoothly to the complete basis set limit. As a consequence, complete basis set extrapolation is not feasible. In light of this problem, I decided to re-evaluate the size scope of the benchmark set being used. In order to facilitate future DFT-based work and probe the issue of basis set convergence of alkali and alkaline earth metals, a benchmark set of small substrates was proposed. This new set is shown in \ref{fig:set2}. The new, small benchmark set was selected to include important functional groups and radicals found biological systems, and one of the most common solvents used in physical organic experiments, acetonitrile.

\begin{scheme}[!htbp]
  \centering
    \includegraphics[width=\textwidth]{figures/set2.eps}
    \caption{Revised benchmark set of small substrates and cations. Note this set consists of all combinations of substrates and metal cations, i.e., there are 35 complexes in the set.}
  \label{fig:set2}
\end{scheme}

\subsection{Metal cation basis set convergence}

\begin{figure}[!htbp]
  \centering
    \includegraphics[width=\textwidth]{figures/pes_metals}
    \caption[Basis set convergence for alkali and alkaline earth-metal cations.]{Basis set convergence of CCSD(T,Full)/aug-pc-$N$ ($N$=1,2,3,4) for alkali and alkaline earth-metal cations. The relative energy of each basis set relative to the aug-pc-1 for each metal. The cardinal number of the aug-pc-$N$ basis sets is $X=N+1$.}
  \label{fig:pes_metals}
\end{figure}

In order to perform complete basis set (CBS) extrapolation, the total energy of a molecule/atom should converge smoothly to the CBS limit.\cite{Truhlar1998} However,~\ref{fig:pes_metals} shows the poor basis set convergence of CCSD(T,Full)/aug-pc$N$ ($N$=1,2,3,4) calculations for alkali and alkaline earth-metals. The energy of each ion relative to the smallest basis set is shown. For \ch{Li^+} the value appears to converge reasonably, however this is expected as there are only 2 electrons in this ion. For all of \ch{Na^+}, \ch{Mg^{2+}}, \ch{K^+}, and \ch{Ca^{2+}}, there appears to be no convergence to the CBS limit as each line continues down linearly. This is problematic as it means that CBS extrapolation would result in a significant degree of uncertainty in the estimated CBS limit total energy.

The poor convergence was thought to be a result of poorly suited basis sets to full core-correlation. However, the same CCSD(T,Full) calculations using the core-correlation (cc-pCV$N$Z) basis sets as show unsatisfactory convergence for \ch{Na^+} and \ch{Mg^{2+}} (See Appendix~\ref{ap:hat}, ~\ref{fig:ap_pes_metals}). Therefore, I was tasked with finding a method which would give results which best approximate alkali and alkaline metal binding at the complete basis set limit. I decided to use an explicitly correlated CCSD(T) treatment known as ``F12$^*$'' to more rapidly approach the CBS limit.\cite{Tenno2012} I tested both the core-correlation consistent basis set developed for used with explicitly correlation (cc-pCV$X$Z-F12),\cite{Peterson2008} and the Ahlrich basis sets (Def2-SVP,-TZVPPD, and -QZVPPD).\cite{Rappoport2010} Both these basis sets combined with the CCSD(T,Full)-F12$^*$ methodology gave satisfactory convergence to the CBS limit for the sodium and magnesium ion (See Appendix ~\ref{ap:hat},~\ref{fig:ap_metals_explicit}) for the convergence of all metal ions calculated with the CCSD(T,Full-F12$^*$/Def2-QZVPPD method). Given that Def2-QZVPPD is available for almost every atom on the periodic table, and the observed convergence to the CBS limit, this basis set was selected for benchmark quality binding energies.

To the best of my knowledge, there is no precedent for extrapolating the Ahlrich basis sets, thus the final benchmark energies are at the CCSD(T)-F12$^*$/Def2-QZVPPD level of theory without extrapolation. The convergence of the total energies of the cations can be estimated as the sum of the experimental ionization energies of the ions. These results are listed in~\ref{tab:metal-energy}. The calculated values are in too high (i.e., not at the CBS limit), and deviate from experiment from 0.16 and 0.66 AU (4.4--18 eV). Deviations of this magnitude are rather significant, but can likely be accounted for by cumulative experimental error, as the experimental ionization energies range from 4--5500 eV. Therefore, the calculated binding energies herein are likely the best available approximation to the gas-phase metal-substrate binding energy.

\begin{table}[!htbp]
  \caption[Total energy of alkali and alkaline earth-metal cations.]{Total energy of alkali and alkaline earth-metal cations from experimental ionization energies\cite{CRC2016} (Expt.) and calculated (Calc.) at the CCSD(T,Full)-F12$^*$/Def2-QZVPPD level of theory. All values are in units of AU.}
  \label{tab:metal-energy}
  \begin{tabular}{l c c}
    \textbf{Ion} & \textbf{Expt.} & \textbf{Calc.} \\
    \hline
    \ch{Li^+} & -7.47798 & -7.27983 \\
    \ch{Na^+} & -162.43089 & -162.24203 \\
    \ch{Mg^{2+}} & -200.32523 & -199.49171 \\
    \ch{K^+} & -601.93332 & -601.77381 \\
    \ch{Ca^{2+}} & -680.19158 & -679.53065
  \end{tabular}
\end{table}

\subsection{High level results and evaluation of various density-functional theory based methods}

\ref{tab:ccsd-metal} lists the benchmark binding energy values calculated at the CCSD(T,Full)-F12$^*$/Def2-QZVPPD//LC-$\omega$PBE-D3(BJ)/6-311+G(3df,3pd) level of theory. Some general trends are that alkaline earth-metals bind more strongly than alkali earth-metals. In general the order of binding follows the Lewis acidity of the metal ions: \ch{Mg^{2+}} $>$ \ch{Ca^{2+}} $>$ \ch{Li^+} $>$ \ch{Na^+} $>$ \ch{K^+}. Also, the metals all appear to bind most strongly to the amidic oxygen-centre, reflecting the higher Lewis basicity. All the metals also bind weakest to the oxygen-centred radicals, with greater binding to \ch{HOO^.} as compared to \ch{HO^.}.

\begin{table}[!htbp]
  \caption[Benchmark gas-phase binding energies of alkali and alkaline earth-metals with small organic substrates and radicals.]{Benchmark gas-phase binding energies of alkali and alkaline earth-metals with small organic substrates and radicals. Values are calculated at the CCSD(T,Full)-F12$^*$/Def2-QZVPPD//LC-$\omega$PBE-D3(BJ)/6-311+G(3df,3pd) level of theory. All values are in \kcalmol.}\label{tab:ccsd-metal}
  \begin{tabular}{l c c c c c}
            &\ch{Li^+}&\ch{Na^+}&\ch{Mg^{2+}}&\ch{K^+}&\ch{Ca^{2+}}\\
    \hline
    \ch{H2O}    & -34.7 &  -24.4 &  -82.0  &  -17.8 &  -56.8 \\
    \ch{NH3}    & -39.9 &  -28.2 &  -98.1  &  -19.8 &  -65.3 \\
    MeCN        & -44.4 &  -33.0 &  -113.1 &  -24.9 &  -80.7 \\
    Formamide   & -50.7 &  -36.9 &  -128.2 &  -28.5 &  -96.1 \\
    Formic acid & -38.4 &  -27.0 &  -101.9 &  -20.0 &  -72.6 \\
    \ch{HO^.}   & -21.3 &  -16.8 &  -57.0  &  -12.4 &  -40.7 \\
    \ch{HOO^.}  & -27.1 &  -19.1 &  -72.2  &  -13.9 &  -49.0
  \end{tabular}
\end{table}

Next, 31 DFT-based methods combined with a moderate basis set (6-31+G(2d,2p)) and a large basis set (6-311+G(3df,3pd)) were tested for their ability to estimate the binding energy between metal cations and substrates. The mean absolute/signed errors (MAE/MSE) and maximum and minimum errors for each method are listed in~\ref{tab:dft-metal}.

\setlength\LTleft{-1cm}
\begin{longtable}[!htbp]{m{4cm} c c | c c}
\caption[Evaluation of DFT-based methods for alkali and alkaline metal binding to organic substrates and radicals.]{Evaluation of DFT-based methods for alkali and alkaline metal binding to organic substrates and radicals. All values are in \kcalmol. Negative values indicate under-binding.}
\label{tab:dft-metal}\\
\textbf{Method}&\textbf{MAE/MSE}&\textbf{Max./Min}&\textbf{MAE/MSE}&\textbf{Max./Min.}\\
\hline
 & \multicolumn{2}{c|}{6-311+G(3df,3pd)} & \multicolumn{2}{c}{6-31+G(2d,2p)}\\
B3\cite{Becke1993}LYP\cite{Lee1988} &  1.49/1.35 &  5.12/-0.57  &  1.59/-0.17 &  4.67/-7.28 \\
B3P86\cite{Perdew1986} &  0.94/0.47 &  3.87/-0.96  &  1.36/-1.08 &  1.99/-7.38 \\
B3PW91\cite{Perdew1991}  &  0.95/-0.14 &  2.74/-1.64  &  1.89/-1.70 &  1.47/-8.76 \\
BH+H\cite{Becke1993a}LYP &  1.89/1.84 &  5.29/-0.59  &  1.93/0.63 &  4.65/-5.64 \\
B\cite{Becke1988}LYP  &  1.60/1.07 &  5.56/-1.51  &  1.88/-0.75 &  5.30/-8.80 \\
BMK\cite{Boese2004}   &  0.90/-0.70 &  1.13/-2.40  &  1.98/-1.93 &  0.86/-8.75 \\
BP86              &  1.63/-0.25 &  4.61/-3.21  &  2.27/-2.14 &  1.55/-9.38 \\
CAM-B3LYP\cite{Yanai2004} &  2.40/2.40 &  6.25/ 0.21  &  1.98/1.04 &  5.82/-5.25 \\
LC-$\omega$PBE\cite{Vydrov2006, Vydrov2006a} &  0.78/0.58 &  2.95/-0.73  &  1.34/-0.74 &  2.19/-8.00 \\
M05-2X\cite{Zhao2006} &  1.11/1.11 &  3.21/ 0.15  &  1.24/-0.17 &  2.55/-5.75 \\
M06\cite{Zhao2008}    &  1.05/-0.62 &  2.36/-4.83  &  1.83/-1.63 &  1.76/-9.03 \\
M06-2X\cite{Zhao2008} &  1.13/1.13 &  3.68/ 0.11  &  1.26/-0.07 &  3.00/-6.63 \\
M06L\cite{Zhao2006b} &  1.52/-1.14 &  2.64/-6.94  &  2.55/-2.48 &  1.21/-11.2 \\
MOHLYP\cite{Schultz2005} &  2.30/-2.02 &  1.52/-5.40  &  4.04/-3.96 &  0.89/-15.2 \\
PBE0\cite{Adamo1999, Ernzerhof1999} &  1.22/1.18 &  4.19/-0.31  &  1.25/-0.34 &  3.30/-7.25 \\
PBE\cite{Perdew1996} &  1.70/1.46 &  6.09/-0.87  &  1.58/-0.46 &  4.68/-8.15 \\
TPSS\cite{Tao2003}   &  1.38/0.95 &  4.88/-1.12  &  1.60/-0.98 &  2.91/-8.12 \\
B97\cite{Becke1997}D3\cite{Grimme2010} &  1.50/0.47 &  5.94/-2.19  &  1.69/-1.37 &  2.67/-8.41 \\
$\omega$B97\cite{Chai2008a} &  0.61/0.23 &  2.13/-1.72  &  1.41/-0.97 &  1.78/-7.57 \\
$\omega$B97XD\cite{Chai2008} &  1.12/-0.94 &  1.02/-4.52  &  2.24/-2.21 &  0.65/-8.64 \\
HSEH1PBE\cite{Heyd2004, Heyd2005} &  1.30/1.29 &  4.28/-0.16  &  1.23/-0.23 &  3.45/-6.95 \\
B3LYP-D3(BJ)\cite{Grimme2010, Johnson2006} &  2.86/2.86 &  7.50/ 0.34  &  1.92/1.34 &  7.05/-4.33 \\
BLYP-D3(BJ)       &  2.89/2.88 &  8.40/-0.14  &  1.83/1.06 &  8.14/-5.30 \\
B3PW91-D3(BJ)     &  1.47/1.40 &  5.81/-0.44  &  1.02/-0.14 &  3.88/-5.69 \\
BMK-D3(BJ)        &  1.03/0.80 &  4.05/-1.06  &  1.02/-0.43 &  2.03/-5.49 \\
BP86-D3(BJ)       &  1.77/1.29 &  7.78/-1.05  &  1.26/-0.60 &  3.96/-6.19 \\
CAM-B3LYP-D3(BJ)  &  3.19/3.19 &  7.50/ 0.78  &  2.27/1.82 &  7.07/-3.54 \\
LC-$\omega$PDE-D3(BJ)& 1.47/1.46 & 4.07/-0.06 &  1.33/0.14 &  3.30/-6.15 \\
PBE0-D3(BJ)       &  1.92/1.92 &  5.37/ 0.20  &  1.33/0.40 &  4.47/-5.74 \\
PBE-D3(BJ)        &  2.24/2.23 &  7.57/-0.22  &  1.44/0.30 &  5.90/-6.67 \\
TPSS-D3(BJ)       &  2.03/1.99 &  6.93/-0.30  &  1.25/0.06 &  4.56/-6.03
\end{longtable}
\setlength\LTleft{0cm}

Given the magnitude of the binding energies, the overall agreement between the benchmark values and the DFT-based method values with both moderate and large basis sets is very good. Interestingly, the application of the empirical D3(BJ) dispersion correction systematically decreases agreement with benchmark values as it increases over-binding. Also, going from moderate to large basis sets systematically increases the predicted binding energies, as indicated by an increase in MSE across the board. I chose three of the best performing methods (BMK-D3(BJ), TPSS-D3(BJ), and M05-2X) and performed geometry optimizations with moderate basis sets on the benchmark structures to determine if the choice of method would significantly impact the minimum energy bound structure. These results are listed in~\ref{tab:ccsd-metal-opt}.

\begin{table}
  \caption[Comparison of single point and relaxed binding energies for alkali and alkaline metal binding with DFT-based methods.]{Comparison of single point (SP on benchmark structure) and relaxed (optimized with method) binding energies for alkali and alkaline metal binding with DFT-based methods and 6-31+G(2d,2p) basis sets. Mean absolute error (MAE) values are in \kcalmol\ and average root mean squared deviation (RMSD)\bibnote{Root mean square deviation was calculated using the Kabsch algorithm\protect\cite{Kabsch1976} as implemented in the rmsd package available on GitHub (Calculate RMSD for two XYZ structures, GitHub, http://github.com/charnley/rmsd, accessed Nov. 18, 2016)} of geometry are in \AA.}\label{tab:ccsd-metal-opt}
  \begin{tabular}{l c c c}
    Method & MAE(SP) & MAE(Relaxed) & Average RMSD \\
    \hline
    BMK-D3(BJ) & 1.02 & 1.24 & 0.012 \\
    M05-2X & 1.24 & 1.17 & 0.020 \\
    TPSS-D3(BJ) & 1.25 & 1.21 & 0.026 \\
  \end{tabular}
\end{table}

For all the three methods tested, the average of the root mean square deviations from benchmark structures are very small (0.012--0.026 \AA). For BMK-D3(BJ), re-optimization of the structures results in a slight increase in MAE, while for TPSS-D3(BJ) and M05-2X, the opposite is true. As a whole, it seems that DFT-based methods are capable of capturing alkali and alkaline metal ion binding with organic substrates and molecules. M05-2X appears to be one of the best performing DFT-based methods. Additionally, M05-2X is recommended by the QM-ORSA\cite{Galano2013} method, which outlines best principles for calculating accurate HAT rate constants in solution. And finally, M05-2X was previously used by our group in the study of the HAT reaction between DMSO and \bno,\cite{vanSanten2016} thus I have selected this method for further study of the effects of alkali and alkaline earth metal cations on the barrier heights of HAT reactions. Note also that M05-2X is a hybrid density-functional with 56\% HF-exchange, and thus should not suffer from delocalization error.

\section{Exploring the nature of metal cation substrate interactions}

The first step to understanding the effect non-redox active metal have on HAT reaction barrier heights is investigating the nature of the binding interaction. \ref{fig:pes-dma-na}A,B show the potential energy surfaces (PESs) of the binding of sodium ion, and sodium chloride to DMA, respectively. There are three surfaces in each plot representing the same potential energy surface in the gas-phase (black circles), and in a continuum solvent field of (grey squares) or water (white circles). For both sodium ion and sodium chloride, the gas-phase PES demonstrates severe over-binding with respect to a realistic system. This is indicated by a much deeper well than both solvents and a long range interaction that does not tail off within a reasonable distance. This simply underscores the importance of including solvent effects in studying the effects of metal cations. Intestingly, for the effects of differnt solvent appears to be quite small. In both cases the difference between the mininum of the water PES is about 2.5 \kcalmol, while the differences in range of interaction is neglible. The small differences in binding interactions indicate that the effects as measured in MeCN should also apply to the more biologically relevant aqueous system. Furthermore, Ingold and Litwinienko have shown that \ch{C-H} abstraction by oxygen-centred radicals does not depend strongly on solvent.\cite{Litwinienko200}

For both the ion and the salt of sodium, the binding interaction approaches zero well before 5 \AA, or the size of the first solvation shell of the sodium ion.\cite{Degreve1996} This result is consistent with literature that studies the Hofmeister series, where it has been shown that biologically relavant cations are only able to influence their immediate solvation shell.\cite{Omta2003, Funker2011} Furthermore, \citet{Heyda2009} utilized molecular dynamics simulations of $N$-methylacetamide in aqueous solutions of NaCl, NaBr, KCl, and KBr to obtained radial distribution functions (RDFs). The RDFS are in agree with the the calculated PESs in that the most probable distance to find \ch{Na^+} from the amidic oxygen-centre is at about 2--2.5 \AA separation. Heyda et al. also showed that \ch{Na^+} preferentially binds with the amide as compared to \ch{Br^+}, and that the nature of the halide counteranion is unimportant in the interaction. From these results it is possible an important conclusion: The use of \ch{Cl-} in the calculations should reasonable reflect the trends observed by Salamone et al. with \ch{ClO4-} and \ch{OTf-}.

\begin{figure}[!htbp]
\centering
\vspace{1.0cm}
\hspace*{-1.8cm}
\begin{minipage}{8cm}
  \centering
  \begin{overpic}[width=\textwidth]{figures/pes_dma_na}
  \put(5,70) {\large\textbf{A.}}
\end{overpic}
\end{minipage}%
\begin{minipage}{8cm}
  \centering
  \begin{overpic}[width=\textwidth]{figures/pes_dma_nacl}
  \put(5,70) {\large\textbf{B.}}
\end{overpic}
\end{minipage}
\caption[Potential energy surface of binding energy between DMA and sodium cation and sodium chloride.]{Potential energy surface of binding energy between DMA and \textbf{A} sodium cation and \textbf{B} sodium chloride as a function of O-Na interaction distance (\AA). The black line and points represent gas-phase results, the grey squares and line is in continuum MeCN solvent, and white circles and dashed line is in continuum water solvent. Calculated as a rigid scan from the M05-2X/6-31+G$^{**}$ minimized complex structure at the M05-2X/6-311+G(2d,2p) level of theory with the SMD solvent model.}
\label{fig:pes-dma-na}
\end{figure}

\ref{fig:pes-dma-mg}A,B show the PESs of the binding of magnesium ion, and magnesium chloride to DMA, respectively. As is the case for \ch{Na^+}, the gas-phase PES of\ch{Mg^{2+}} is extremely over-bound, so much so infact, that the interation does not approach zero. At an O--Mg distance of 12 \AA, there is calculated binding interaction of -48.5 \kcalmol, which is actually greater than at 6 \AA by about 5 \kcalmol. The unphysical behaviour is a prime example of a failing of DFT-based methods. Here, the DFT calculations are unable to localize the charge properly due to delocalization error,\cite{Cohen2008} even with the use of a high-percentage HF hybrid density functional. The localization of charges was recently described by \citet{Cheng2016} as a widespread failing of every DFT-based method they tested. The reason this is a problem with \ch{Mg^{2+}} and not \ch{Na^+} has to do with the ionization potentials (IP) of the metals with respect to that of DMA. The expermental IP\cite{Slifkin1967, Baldwin1977, CRC2016} of DMA is 8.8--9.2 eV, the first IP of Na is 5.1 eV, and the second IP of Mg is 15.0 eV (calculated with M05-2X/6-311+G(2d,2p) = 8.9, 5.0, and 14.9 eV, respectively). Then, the ionization of DMA by \ch{Na^+} and \ch{Mg^{2+}} can be described as:

\begin{align*}
\ch{DMA + Mg^2+ -> DMA^+ + Mg^+} \quad  &\Delta E = -6 \mathrm{eV} \\
\ch{DMA + Na^+ -> DMA^+ + Na} \quad &\Delta E = +4 \mathrm{eV}
\end{align*}

The ionization of DMA by \ch{Mg^2+} is favourable by about 6 eV, and unfavourable by \ch{Na+} by about 4 eV. Therefore, DFT-based methods will prefer to delocalize the charge between DMA and \ch{Mg^2+}, but not for \ch{Na+}. A possible resolution to this is to used a constrained DFT method which enables one to specify atomic occupancies,\cite{Melander2016} however this technique is not currently available in most common quantum chemical packages.

\begin{figure}[!htbp]
\centering
\vspace{1.0cm}
\hspace*{-1.8cm}
\begin{minipage}{8cm}
  \centering
  \begin{overpic}[width=\textwidth]{figures/pes_dma_mg}
  \put(5,70) {\large\textbf{A.}}
\end{overpic}
\end{minipage}%
\begin{minipage}{8cm}
  \centering
  \begin{overpic}[width=\textwidth]{figures/pes_dma_mgcl2}
  \put(5,70) {\large\textbf{B.}}
\end{overpic}
\end{minipage}
\caption[Potential energy surface of binding energy between DMA and magnesium cation and magnesium chloride.]{Potential energy surface of binding energy between DMA and \textbf{A} magnesium cation and \textbf{B} magnesium chloride as a function of O-Mg interaction distance (\AA). The black line and points represent gas-phase results, the grey squares and line is in continuum MeCN solvent, and white circles and dashed line is in continuum water solvent. Calculated as a rigid scan from the M05-2X/6-31+G$^{**}$ minimized complex structure at the M05-2X/6-311+G(2d,2p) level of theory with the SMD solvent model.}
\label{fig:pes-dma-mg}
\end{figure}

The inclusion of MeCN and water solvent appear at first glance to alleviate this problem. Note however that both PESs cross over zero binding at about 3 \kcalmol\, indicating there is still delocalization. Additionally, for magnesium chloride in the gas-phase, there also appears to be delocalization error occuring, as evident by the PES crossing zero just below 4 \kcalmol. On the other hand, the inclusion of solvent with \ch{MgCl2} give reasonable PESs with a binding interaction of about 25 \kcalmol\ for both water and MeCN, and a tailing off at about 3 \AA. Therefore, the effects of magnesium on HAT reaction barrier may possibly using \ch{MgCl2}.

Next, I performed calculations to determine if the interaction of metal cations systematically increase the bond strengths of \ch{C-H} bonds by decreasing hyperconjugative overlap between neighbouring $\pi$-systems and \ch{C-H} $\sigma^*$ anti-bonding orbitals. The bond dissociation enthalpies (BDEs) and free energies (BDFEs) for several substrates in the presence of \ch{Na^+}, \ch{NaCl}, \ch{Mg^2+}, and \ch{MgCl2} are listed in~\ref{tab:bde-metal}.

\begin{table}[!htbp]
  \caption[Bond dissociation enthalpies(free energies) of DMA, DMSO, and MeCN with and without metal cations.]{Bond dissociation enthalpies(free energies) of DMA, DMSO, and MeCN with and without metal cations calulated at the M05-2X-SMD/6-311+G(2d,2p)//M05-2X/6-31+G$^{**}$ level of theory. ROCBS-QB3 BDEs and BDFEs are included for reference. All values are in \kcalmol.}
  \label{tab:bde-metal}
  \hspace*{-1.5cm}
  \begin{tabular}{l c c c c c c}
    Substrate       & ROCBS-QB3   &    Bare    &\ch{Na+}    &\ch{NaCl}  &\ch{Mg^2+}&\ch{MgCl2}  \\
    \hline
    DMA (acetyl)    & 99.5(91.3)  & 98.5(90.3) & 97.8(91.0) & 98.4(90.4) &  97.4(88.7)  &  97.8(90.2)   \\
    DMA (cis)       & 93.9(86.0)  & 92.2(84.5) & 93.2(85.8) & 94.0(89.4) &  138.3(129.3) &  93.5(86.3)   \\
    DMA (trans)     & 92.3(84.3)  & 91.6(83.4) & 92.8(85.5) & 92.6(86.1) &  137.3(128.8) &  93.3(86.3)   \\
    DMSO            & 102.2(93.6) & 103.4(94.7)& 104.4(95.5)& 103.7(97.4)&  106.7(97.9) &  105.5(96.1)  \\
    MeCN            & 96.6(88.3)  & 97.4(89.1) & 98.3(89.9) & 98.1(89.8) &  99.5(91.2)  &
  \end{tabular}
\end{table}

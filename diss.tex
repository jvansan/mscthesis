\documentclass[msc,oneside]{ubcthesis}%msc, phd, masc, ma, or meng

% ================================================================================
% CHANGE THE FOLLOWING ACCORDING TO YOUR PROGRAM/THESIS
% ================================================================================
\institution{The University Of British Columbia}
\faculty{THE COLLEGE OF GRADUATE STUDIES}
\institutionaddress{Okanagan}

% For an Honours thesis, use \documentclasss[msc,oneside]{ubcthesis} above and
% uncomment and modify the next line:
%\degreetitle{B.Sc. Computer Science Honours}

\title{A Computational Examination of the Effects of Alkali and
  Alkaline Earth Metals on Hydrogen Atom Abstraction Reactions by
  Oxygen Centered Radicals.}
%\subtitle{With a Subtitle}
\author{Jeffrey A. van Santen} % Name as on Diplome
\copyrightyear{2017}
\submitdate{June 2017} % date of approved thesis
\program{Chemistry}%or Mathematics, or Interdisciplinary Studies
\previousdegree{B.Sc. Hons. Chemistry, The University of British Columbia, 2015}


% ================================================================================


\usepackage{ubcostyle} %loads packages

% ===================================================================
% CHANGE THE FOLLOWING COMMANDS ACCORDING TO YOUR NEEDS
% ===================================================================
\newcommand{\R}{\mathbb{R}}   %real number
\newcommand{\Z}{\mathbb{Z}}   %integers
\newcommand{\C}{\mathbb{C}}   %complex numbers

\newcommand{\dom}{\operatorname{dom}}
\providecommand{\TT}[1]{\Theta\left(#1\right)} % big-Theta
\providecommand{\OO}[1]{\mathcal{O}\left(#1\right)} % big-Oh
% ===================================================================

%Uncomment the next line if there are more than one appendix
%\renewcommand*\appendixname{Appendices}

\begin{document}

% This starts numbering in Roman numerals as required for the thesis
% style.
\frontmatter                    % Mandatory

% The order of the following components should be preserved.  The order
% listed here is the order currently required by FoGS.
\maketitle                      % Mandatory

\makeatletter

The undersigned certify that they have read, and recommend to the
College of Graduate Studies for acceptance, a thesis entitled: {\sc
  \@title } submitted by {\sc \@author} in partial fulfilment of the
requirements of the degree of \@degreetitle \makeatother

\newlength{\linespace}
\setlength{\linespace}{.75cm} %change .75cm to .5cm for smaller space
                              %between signatures or 1cm if less
                              %people have to sign
\vspace{\linespace}\smaller
%%%%%%%%%%%%%%%%%%%%%%%%%%%%%%%%%%%%%%%%%%%%%%%%%%%%%%%%%%%%%%%%%%%%%%%%%%%%%%%%%%
% UPDATE THE FOLLOWING AS PER YOUR THESIS COMMITTEE

\noindent\underline{\hspace{30em}} \\
Supervisor, Professor (please print name and faculty/school above the line)

\vspace{\linespace}

\noindent\underline{\hspace{30em}} \\
Supervisory Committee Member, Professor (please print name and faculty/school above the line)

\vspace{\linespace}

\noindent\underline{\hspace{30em}} \\
Supervisory Committee Member, Professor (please print name and faculty/school above the line)

\vspace{\linespace}

\noindent\underline{\hspace{30em}} \\
University Examiner, Professor (please print name and faculty/school above the line)

\vspace{\linespace}

\noindent\underline{\hspace{30em}} \\
External Examiner, Professor (please print name and faculty/school above the line)

\vspace{\linespace}

\noindent\underline{\hspace{30em}} \\
(Date Submitted to Grad Studies)

\vspace{\linespace}

Additional Committee Members include:

\vspace{\linespace}

\noindent\underline{\hspace{30em}} \\
(please print name and faculty/school above the line)

\vspace{\linespace}

\noindent\underline{\hspace{30em}} \\
(please print name and faculty/school above the line)

%%%%%%%%%%%%%%%%%%%%%%%%%%%%%%%%%%%%%%%%%%%%%%%%%%%%%%%%%%%%%%%%%%%%%%%%%%%%%%%%%% END COMMITTEE PAGE
\normalsize

%%%% ABSTRACT
\newpage
% !TEX root = diss.tex

% Abstract
\begin{abstract}                % Mandatory -  maximum 350 words

\begin{doublespace}

Hydrogen atom transfer (HAT) reactions are a fundamental step in many biological
processes, but can initiate the free-radical induced oxidation of cellular
components. Although HAT reactions appear fundamentally elementary, there are
many poorly understood factors that influence HAT. In this thesis, three aspects
of HAT reactivity are investigated using quantum chemical techniques.

First, the importance of pre-reaction complex formation in considering the
kinetics of HAT reactions were investigated. For a set of nearly-thermoneutral
HAT reactions involving oxygen-centred radicals, the relationship between
pre-reaction complex non-covalent binding energies and Arrhenius pre-exponential
factors (A-factors) was investigated. It is demonstrated that for HAT reactions
that take place through similar mechanisms, there is a strong correlation
between pre-reaction complex binding energies and A-factors. This suggests that
non-covalent interactions may directly affect the kinetics of certain HAT
reactions.

Next, the relationship between bond dissociation energies (BDEs) and reaction
rates for abstraction of a hydrogen from a \ch{C-H} bond by the \cumo\ radical
are investigated in the context of the Bell-Evans-Polanyi (BEP) principle. The
applicability of the BEP principle is examined by exploring a hypothesis: If the
BEP principle is a valid linear free-energy relationship, there should exist two
linear relationships for BDE against the logarithm of HAT rate constant, one for
incipient radicals that are allylic or benzylic, and one for alkyl radicals. It
is demonstrated that there is a reasonably strong correlation for
allylic/benzylic \ch{C-H} bonds, but not for alkyl ones. The BEP principle
should not be used for quantitative prediction, but remains useful as a
conceptual framework.

Finally, the effect of non-redox active metal cations on HAT reactions involving
small models for proteins and oxygen-centred radicals is studied. Previous
experimental evidence demonstrated that Lewis acid-base interactions between
metal cations and substrates can inhibit HAT reactions, and that the cations may
serve as a form of chemo-protection in biological systems. The results herein
demonstrate that metal-substrate interactions can deactivate certain \ch{C—H}
bonds. Metal-radical interactions may promote HAT reactions. On the
basis of these limited results, non-redox active metal cations might not act as
natural chemo-protective agents.

\end{doublespace}

\end{abstract}


%%%% PREFACE
\newpage
% Thesis may or may not need a Preface
\chapter{Preface}
Preface stuff

If any part of your thesis was co-written, you must include a
Co-Author\-ship statement. Also indicate if part of the thesis was published with the reference.


%%% TABLES
\newpage
\phantomsection \label{tableofcontent}%set anchor at right location
\addcontentsline{toc}{chapter}{\contentsname}
\tableofcontents                % Mandatory: generate toc
\newpage
\phantomsection \label{listoftab}%set anchor at right location
\addcontentsline{toc}{chapter}{\listtablename}
\listoftables                   % Mandatory if thesis has tables
\newpage
\phantomsection \label{listoffig}%set anchor at right location
\addcontentsline{toc}{chapter}{\listfigurename}
\listoffigures                  % Mandatory if thesis has figures


%%% ACKNOWLEDGEMENT and DEDICATION
\chapter{Acknowledgements}      % Optional
This is the place to thank professional colleagues and people who have
given you the most help during the course of your graduate work.

\chapter{Dedication} % Optional
The dedication is usually quite short, and is a personal rather than
an academic recognition.  The \emph{Dedication} does not have to be
titled, but it must appear in the table of contents.  If you want to
skip the chapter title but still enter it into the Table of Contents,
use this command \verb|\chapter[Dedication]{}|.


% Any other unusual prefactory material should come here before the
% main body.

% Now regular page numbering begins.
\mainmatter

% Parts are the largest structural units, but are optional.
%%\part{Thesis}

% Chapters are the next main unit.
%%% CHAPTERS
% !TEX root = test.tex

 \chapter{Introduction}

%\section{Introduction}

Radicals are chemical species which tend to be highly reactive due to the
presence of one or more unpaired electrons. Living systems depend on radical
processes as part of normal metabolism but biological molecules, such as
proteins, are susceptible to radical induced damage. Radical induced oxidation
of biomaterials has been implicated in a number of degenerative disease states,
including cancer, Alzheimer's Disease, Parkinson's Disease, and multiple
sclerosis.\cite{Barnham2004,Halliwell2007,Valko2007,Hwang2013,Halliwell2015}

In biological systems, radicals are derived from both endogenous sources, such
as transition metal-ion redox processes and other \emph{in vivo} processes, as
well as exogenous sources, for instance, solar radiation and air pollutants.
Oxygen centred radicals, known as reactive oxygen species (ROSs) in biology, are
particularly important due to the nature of the aerobic respiration. The
radicals of primary concern are the highly reactive hydroxyl radical
(\ch{HO^.}), alkoxyl radicals (\ch{RO^.}), superoxide (\ch{HOO^.}/\ch{O2^{.-}}),
and peroxyl radicals (\ch{ROO^.}).\cite{Halliwell2015} Damage occurs when an ROS
initiates a radical chain reaction through hydrogen atom transfer (HAT),
electron transfer, or addition reactions, leading to rapid propagation. HAT is
the most relevant reaction and is the focus of my work.

Hydrogen atom transfer (HAT) reactions are a fundamental radical chemical
transformation which has been studied for over a
century.\cite{Kochi1973,Parsons2000} At the macroscopic level, HAT reactions
which involve oxygen centred radicals and non-radical organic substrates are
reasonably well characterised: the effects of bulk solvent are well
understood.\cite{Litwinienko2007} However, the roles of substrate-radical and
substrate-radical-medium interactions at the microscopic (molecular) level
continue to be relatively poorly understood.

Recent work from our group, in collaboration with colleagues at University of
Rome Tor Vergata, has focused on the importance of substrate-radical
interactions. Specifically, it has been shown that the three-dimensional
structures of oxygen centred radicals, as well as the organic substrates,
impacts the nature of the interactions involved in HAT reaction pathways. In our
work, we utilise primarily the \bno and \cumo radicals, which serve as a
convenient proxy to biological oxygen centred radicals. Reaction involving \bno
and \cumo can be easily monitored using highly resolved laser flash photolysis
(LFP) techniques. A combination of theoretical and experimental techniques, have
been used to examine reactions involving \bno and \cumo with a variety of
organic substrates. A detailed discussion of these results shall be reserved for
following chapters, however, a great deal of insight has been gained into the
role of the structural of the radicals and substrates, and resulting
intermolecular interactions.

Recent experimental result show that non-redox active metal cations, which are
found ubiquitously in biological systems, have an inhibitory effect on HAT
reactions involving oxygen centred radicals and substrates which undergo
abstraction from sites adjacent to heteroatoms (e.g. amines, amides, and
ethers). Under various stoichiometric ratios, these metal cations have effects
ranging from full inhibition to partial deactivation of HAT
reactivity.\cite{Salamone2013,Salamone2015metals,Salamone2016} This effect has
been attributed partially to the effects of hyperconjugative overlap. Take for
example tetrahydrofuran (THF), shown in \ref{fig:THF}. Normally, there exists
C-H bond weakening hyperconjugative overlap of electron density from one of the
oxygen lone-pairs and the adjacent C-H $\sigma^*$ anti-bonding orbitals. The
interaction of a metal cation with the oxygen lone-pairs removes electron
density from this interaction, thus increasing the C-H bond strength. As a
results, the reactivity of this bond is decreased, as observed from the
experimentally measured 3.2-fold decrease in the rate constant for HAT with
\cumo in acetonitrile from 6.65 \E{7} \Ms to 7.0 \E{7} \Ms in the presence of
1.0 M \ch{Mg(ClO4)2}.\cite{Salamone2013}

\begin{figure}[htb]
  \centering
  \includegraphics[width=0.65\textwidth]{figures/THF}
  \caption[Hyperconjugative overlap in tetrahydrofuran and the effect of non-redox active metal cations.]
  {Hyperconjugative overlap in tetrahydrofuran and the effect of non-redox active metal cations. The metal cation acts accepts electron density from the heteroatom lone pair, reducing overlap with the C-H $\sigma^*$ anti-bonding orbital and increasing the C-H bond strength.}
  \label{fig:THF}
\end{figure}

The nature of the interactions between non-redox active metal cations and
organic substrates is poorly understood. The primary goal of this thesis is to
understand the fundamental physico-chemical properties which lead to the
experimentally observed trends in reactivity. This problem is explored in
\iffalse\ref{ch:hat}\fi Chapter 5 \jnote{update}. The observed effects have led
us to hypothesise that the presence of non-redox active metal cations have a
chemoprotective effect against the radical induced oxidation of biomaterials
such as proteins.

In addition \jnote{Add details of Chs 3 and 5}

% The effects of metal cation substrate interactions occur for unusual
% stoichiometric ratios, as illustrated by \ref{fig:PMP}, which shows that the
% observed rate constant for addition of a tertiary amine (PMP in this case) , is
% steadily consistent with the dominant \cumo degradation pathway of
% $\beta$-scission at $\approx$ 7\E{5} s$^-1$ up until a ratio of 1.5:1 of metal
% cation to amine.
%
% \begin{figure}[htb]
%   \centering
%   \includegraphics[width=0.65\textwidth]{figures/PMP}
%   \caption{Plot of observed rate constant versus concentration of PMP}
%   \label{fig:PMP}
% \end{figure}


%% SAMPLE CHAPTERS
%\chapter{Sample Content Using Mathematical Notations}

\section{Facts and theorems}

If we use a well established fact or theorem\index{theorem}, we state
it with a citation in the paragraph title of the fact or theorem. The
following is from a well known textbook.\footnote{Note that in this
  definition, we use the \texttt{gls} command for the newly used
  symbols.}

\begin{fact}\cite[Theorem~IV.2.4.2]{Hiriart-Urruty:1993-ConvexAnalysis}\label{def:marginalfunc}
  Define the \emph{marginal function} $\gamma$ associated with
  $g:\R^n\times\R^m\rightarrow \R\cup \{+\infty\}$ by
  $z\mapsto \gamma(z):=\inf_x g(x,z)$. If $g$ is a proper convex
  function and is bounded below on the set $\R^n \times \{z\}$ for all
  $z$, then $\gamma$ is convex.
  \glsadd{Real}\glsadd{Rvec}\glsadd{Cart}\glsadd{Infinity}\glsadd{Infimum}
\end{fact}

\section{Propositions and lemmas}

Here is a lemma followed by its proof.
\[
D =\left\{ (x,\lambda)\in \R^d \times \R^+ : \frac{x}{\lambda} \in C\right\}.
\]
\glsadd{Rplus}

\begin{lemma}
  Assume $C$ is a nonempty closed convex set. Then the set $D$ is a
  nonempty closed convex cone.
\end{lemma}

\begin{proof}
  The fact that $D$ is nonempty and closed follows from $C$ being
  non\-empty and closed. One can check directly that $D$ is a cone....

Hence $D$ is convex.
\end{proof}
Make sure that the qed symbol is always on the last line of the
proof. If the last line is an equation, you can enforce the qed on the
same line with the \texttt{qedhere} command.

For citations, please use BibTex. A sample article to verify
formatting and style is \cite{Bauschke:2007-PA02}. Use the
bibliography style \texttt{ubco}, which is basic \texttt{alphaurl}
style with inline links enabled. Please compile multiple times when
generating the references. The last entry in a reference are the back
references to the pages with the citation. They need an additional
compilation, once the bibtex entries are generated.

Note that the bibliography style is discipline dependent so feel free
to use the style adopted by your discipline, for example siam for
mathematics.

%\chapter{Landscape Mode}
The landscape mode allows you to rotate a page through 90 degrees.  It
is generally not a good idea to make the chapter heading landscape,
but it can be useful for long tables etc.

\begin{landscape}
  This text should appear rotated, allowing for formatting of very
  wide tables etc.  Note that this might only work after you convert
  the \texttt{dvi} file to a postscript (\texttt{ps}) or \texttt{pdf}
  file using \texttt{dvips} or \texttt{dvipdf} etc.
\end{landscape}



% !TEX root = diss.tex

\chapter{Conclusions}

\begin{doublespace}

HAT reactions are amongst the simplest radical chemical transformations. This
can be deceiving, as there are many poorly understood factors that influence
HAT. It is important to develop a full understanding of HAT reactions as they
are a fundamental step in many biochemical processes. The radical-induced
oxidation of biomaterials is often trigged by HAT reactions, and has been
implicated in a number of degenerative disease states. In this thesis, three
aspects of HAT reactivity were explored using quantum chemical techniques.

First, the role of non-covalent binding in the pre-reaction complex of HAT
reactions was investigated. In particular, the relationship between the
calculated pre-reaction complex binding energies and experimentally determined
Arrhenius pre-exponential factors (A-factors) was examined for a series of
thermoneutral or nearly thermoneutral reactions involving the formation and
destruction of oxygen-centred radicals. The interpretation of the results of
this investigation relies on an assumption: the mechanism for formal HAT, which
is normally described as either direct HAT of PCET, exists on a continuum that
is described by Equation~\ref{eq:A-theory}.

It was demonstrated that there may be a correlation between A-factors and
pre-reaction complex binding energies for (nearly) thermoneutral HAT reactions
given that the reactions follow similar reaction mechanisms. This is an
important caveat, as it is clear that binding energies do not serve as a
diagnostic for fully describing all the entropic contributions to A-factors.
Specifically, for a set of ten self-exchange and pseudo-self-exchange reaction,
six out of ten of the reactions studied share similar reaction mechanisms, and
there is a strong correlation ($R^2$ = 0.949) of pre-reaction complex binding
energies with A-factors. In order to apply this analysis to further systems,
future work should aim at determining a quantitative diagnostic for the
contributions of PCET and direct HAT to the overall hydrogen transfer reaction
mechanism.

Next, the validity of the Bell-Evans-Polanyi (BEP) principle was investigated by
analyzing a series of HAT reactions in which a hydrogen atom is abstracted from
a carbon centre by the \cumo\ radical. A hypothesis on the basis of
group-additivity states that if the BEP principle is valid, there should exist
two linear relationships for \ch{C-H} bonds, namely, one in which the incipient
radical is delocalized into a $\pi$-system (benzylic or allylic), and the other
in which the remaining alkyl radicals are largely localized. Detailed analysis
of experimentally determined $\log(k_H/n)$ plotted against theoretically
determined \ch{C-H} BDEs demonstrated that there was reasonable correlation
($R^2$ = 0.889) in the case of allylic/benzylic substrates, however the
correlation was not strong for all other alkyl substrates ($R^2$ = 0.641).
Breaking the larger group of alkyl substrates into smaller chemical groups
suggested that specific groups of alkyl containing species (e.g. cyclic alkanes
and alkyl group with heteroatomic neighbours) may demonstrate linear
relationships, however more data are needed to support this.

To explore this further, calculations were performed to determine the structures
of relevant transition state complexes and reaction barrier heights. It was
demonstrated that HAT reactions involving \cumo\ likely proceed through a
mechanism that is dominated by direct HAT. As a result differences in
mechanistic details may be ruled as a factor contributing to the observed poor
correlation. Decomposition of the free energy barrier heights revealed that HAT
reactions involving \cumo\ are entropy-controlled ($-T\Delta S^\ddagger >>
\Delta H^\ddagger$), and non-isoentropic. It is established in the literature
that entropy-controlled processes do not follow LFERs
consistently.\cite{Exner1973} Therefore, while these results do not invalidate
the BEP principle as a LFER, it is apparent that the model is an
over-simplification of the complexity associated with HAT reactions involving
\cumo. This is likely to be the case for many HAT reactions. As a result, the
BEP principle should not be used to as a quantitative prediction tool, but
should remain a conceptual framework to qualitatively describe changes in
reaction rates.

Finally, recent experimental evidence demonstrated that non-redox active metal
cations can have an inhibitory effect on HAT reactions involving oxygen-centred
radicals. As these metals are found ubiquitously in biological systems, it was
suggested that this may be a form of chemo-protection against radical induced
oxidation of biomaterials, in particular proteins. It was previously reported
that the mechanism for this inhibition relies on \ch{C-H} bond deactivation,
which may be the result of a metal cation binding to a substrate. Herein, I
sought to determine the exact mechanism by which this occurs. It was previously
hypothesized that the metal bound to the substrate via an electron rich centre,
such as the oxygen of a carbonyl. Then, electron density is withdrawn from a
neighbouring \ch{C-H} $\sigma^*$ orbital, which is normally populated through
hyperconjugation. Withdrawing electron density from the \ch{C-H} $\sigma^*$
orbital strengthens the bond, and thus the HAT reaction barrier height may
increase, as per the BEP principle.

Two amides, DMA and DIA, were used as small models for protein systems. The
calculated \ch{C-H} BDEs of these substrates both with and without metal cations
demonstrate that specific metal-substrate interactions can increase the
effective \ch{C-H} BDEs in the parent molecule. The calculated $N$-methyl
\ch{C-H} BDEs of DMA increase by an average of 1.4 \kcalmol\ upon complexation
of \ch{NaCl}. This is however complicated by the fact that the product radical
can also interact with the metal cation, which can either increase or decrease
the BDE. For example, the acetyl \ch{C-H} BDE of DMA decreases by 0.1 \kcalmol\
upon complexation of \ch{NaCl}. As a result of metal-radical interaction, BDEs
may not properly reflect the effect metal cations have on HAT barrier heights.

The hypothesis that \ch{C-H} bond strengths are increased by interactions of the
metal cation with the substrate does not account for interactions of that metal
with species other than the substrate. This became abundantly obvious in
calculating the reaction barrier heights including \ch{NaCl} of DMA and DIA with
the radicals \cumo\ and \bno. Specifically, the predicted changes in barrier
heights as a result of binding of \ch{NaCl} to the substrate were found to both
increase and decrease. For the HAT reaction of DMA and \cumo\ including
\ch{NaCl}, the predicted reaction barrier height for abstraction from an
$N$-methyl hydrogen trans to the carbonyl increases as there are no interactions
of \ch{NaCl} with \cumo. However, for abstraction from an $N$-methyl cis to the
carbonyl decrease as a result of a charge transfer interaction between \ch{NaCl}
and \cumo. Analysis of the interactions of \ch{NaCl} with both the radical and
extensions of the substrate demonstrated that secondary interaction can either
stabilize or destabilize the TS complex.

The aforementioned hypothesis does not account for secondary interactions,
however it was based on experimental results which demonstrate that alkali and
alkaline earth metal salts do inhibit HAT reactions of amides with \cumo. The
results obtained herein do not support the hypothesis. An explanation for this
discrepancy remains unclear, however I have speculated that this may be a result
of problems with the model chosen. Specifically, the experimental data suggests
that stoichiometric ratios of substrates and metal cations are somehow related
to the inhibitory effect. No multiple coordinating stoichiometries were
considered herein.

Furthermore, the complexity of protein systems increases the likelihood that
alkali and alkaline earth metal cations will interact not only with a single
peptide moiety in a protein environment. On the basis of these limited results,
it is possible that these metal cations may not serve as natural
chemo-protective agents against radical induced oxidation in biological systems.

There is one important theme that arises from all the results in this thesis:
HAT reactions are deceptively simple. In each chapter, an attempt was made to
make a generalization about the physico-chemical properties of HAT reactions. In
each case, some properties of HAT reactivity were elucidated, however further
questions arose. This work sets the stage for further research into three
important aspects as HAT reactivity.


\end{doublespace}



%generate numerous entries to test pagestyle in index
%source: http://www.tex.ac.uk/ctan/indexing/makeindex/doc/makeindex.pdf
\index{a}\index{b}\index{c}\index{d}\index{e}
\index{f}\index{g}\index{h}\index{i}\index{j}
\index{k}\index{l}\index{m}\index{n}\index{o}
\index{p}\index{q}\index{r}\index{s}\index{t}

\index{aa}\index{ab}\index{ac}\index{ad}\index{ae}
\index{af}\index{ag}\index{ah}\index{ai}\index{aj}
\index{ak}\index{al}\index{am}\index{an}\index{ao}
\index{ap}\index{aq}\index{ar}\index{as}\index{at}

\index{aa}\index{ab}\index{ac}\index{ad}\index{ae}
\index{af}\index{ag}\index{ah}\index{ai}\index{aj}
\index{ak}\index{al}\index{am}\index{an}\index{ao}
\index{ap}\index{aq}\index{ar}\index{as}\index{at}

\index{aaa}\index{aab}\index{aac}\index{aad}\index{aae}
\index{aaf}\index{aag}\index{aah}\index{aai}\index{aaj}
\index{aak}\index{aal}\index{aam}\index{aan}\index{aao}
\index{aap}\index{aaq}\index{aar}\index{aas}\index{aat}

\index{aaaa}\index{aaab}\index{aaac}\index{aaad}\index{aaae}
\index{aaaf}\index{aaag}\index{aaah}\index{aaai}\index{aaaj}
\index{aaak}\index{aaal}\index{aaam}\index{aaan}\index{aaao}
\index{aaap}\index{aaaq}\index{aaar}\index{aaas}\index{aaat}

\index{aaaa}\index{aaab}\index{aaac}\index{aaad}\index{aaae}
\index{aaaf}\index{aaag}\index{aaah}\index{aaai}\index{aaaj}
\index{aaak}\index{aaal}\index{aaam}\index{aaan}\index{aaao}
\index{aaap}\index{aaaq}\index{aaar}\index{aaas}\index{aaat}

%subcategory
\index{a!sub A}
\index{a!sub B}
\index{a!sub A!sub sub A}
%indexing symbols
\index{a@$\alpha$}



% This file is setup to use a bibtex file sample.bib and uses the
% plain style.  Other styles may be used depending on the conventions
% of your field of study.
%
% Note: the bibliography must come before the appendices.


%change heading ``Chapter 5 Bibliography''->''Bibliography''
\newpage %newpage needed otherwise pagestyle applied to previous chapter. Does not actually create a new page
\pagestyle{fancy}\chead{Bibliography}\rhead{}\cfoot{}\rfoot{\thepage}

%Bibliography style is discipline dependent. Mathematic student can use e.g. SIAM
%\bibliographystyle{achemso}
\bibliographystyle{siam}
\bibliography{biblio}%name of your .bib file

\newpage
\pagestyle{headings}

\addtocontents{toc}{%
\protect\renewcommand*\protect\cftchappresnum{\appendixname~}}

%%\part{Appendix}

\appendix
\addappheadtotoc %uses the current page number when it makes the entry in the ToC
\appendixpage

\addtocontents{toc}{
\setlength{\cftbeforechapskip}{\cftbeforesecskip}
\setlength{\cftchapindent}{\cftsecindent}
\protect\renewcommand{\cftchapfont}{\cftsecfont}
\protect\renewcommand{\protect\cftchapdotsep}{\cftsecdotsep}
}


\chapter{Tables}

Here you can have additional tables. Table captions are always on top.

In order to use publication quality tables, one should use the
guidelines in \cite{Fear:2005manual}. In short, do not use vertical
rules or double rules, units in the column heading (not in the body of
the table), precede decimals with a digit, and do not use ditto
signs. Table \ref{table:food} is according to the guidelines.

For tables, the caption goes on top, for figures, the caption goes on
the bottom. If possible, always position tables and figures at the top
of a page.\footnote{In this case, the chapter heading prevents the
  table from being at the top.} Use the option \verb|tbph| for the
placement.

\begin{table}[tbph]
\centering
\caption{A publication quality table. Very very very very very very very very very very long title.
\label{table:food}}
\begin{tabular}{@{}llr@{}} \toprule
\multicolumn{2}{c}{Item} \\ \cmidrule(r){1-2}
Animal & Description & Price (\$)\\ \midrule
Gnat & per gram & 13.65 \\
& each & 0.01 \\
Gnu & stuffed & 92.50 \\
Emu & stuffed & 33.33 \\
Armadillo & frozen & 8.99 \\ \bottomrule
\end{tabular}
\end{table}

\newpage
And other table materials (I needed to generate two pages for that
appendix to test the formatting of the table of content).

\begin{table}
\caption{Another table}
\end{table}

\begin{table}
\caption{Another table}
\end{table}
\begin{table}
\caption{Another table}
\end{table}
\begin{table}
\caption{Another table}
\end{table}
\begin{table}
\caption{Another table}
\end{table}

\begin{table}
\caption{Another table}
\end{table}
\begin{table}
\caption{Another table}
\end{table}
\begin{table}
\caption{Another table}
\end{table}
\begin{table}
\caption{Another table}
\end{table}
\begin{table}
\caption{Another table}
\end{table}

\chapter{Figures}
Here you can have additional figures. Figure captions are always at the bottom.

\newpage

And other additional figures (again I needed to generate two pages :-).
% Indices come here.


\end{document}
\endinput

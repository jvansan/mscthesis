\chapter{Sample Content Using Mathematical Notations}

\section{Facts and theorems}

If we use a well established fact or theorem\index{theorem}, we state
it with a citation in the paragraph title of the fact or theorem. The
following is from a well known textbook.\footnote{Note that in this
  definition, we use the \texttt{gls} command for the newly used
  symbols.}

\begin{fact}\cite[Theorem~IV.2.4.2]{Hiriart-Urruty:1993-ConvexAnalysis}\label{def:marginalfunc}
  Define the \emph{marginal function} $\gamma$ associated with
  $g:\R^n\times\R^m\rightarrow \R\cup \{+\infty\}$ by
  $z\mapsto \gamma(z):=\inf_x g(x,z)$. If $g$ is a proper convex
  function and is bounded below on the set $\R^n \times \{z\}$ for all
  $z$, then $\gamma$ is convex.
  \glsadd{Real}\glsadd{Rvec}\glsadd{Cart}\glsadd{Infinity}\glsadd{Infimum}
\end{fact}

\section{Propositions and lemmas}

Here is a lemma followed by its proof.
\[
D =\left\{ (x,\lambda)\in \R^d \times \R^+ : \frac{x}{\lambda} \in C\right\}.
\]
\glsadd{Rplus}

\begin{lemma}
  Assume $C$ is a nonempty closed convex set. Then the set $D$ is a
  nonempty closed convex cone.
\end{lemma}

\begin{proof}
  The fact that $D$ is nonempty and closed follows from $C$ being
  non\-empty and closed. One can check directly that $D$ is a cone....

Hence $D$ is convex.
\end{proof}
Make sure that the qed symbol is always on the last line of the
proof. If the last line is an equation, you can enforce the qed on the
same line with the \texttt{qedhere} command.

For citations, please use BibTex. A sample article to verify
formatting and style is \cite{Bauschke:2007-PA02}. Use the
bibliography style \texttt{ubco}, which is basic \texttt{alphaurl}
style with inline links enabled. Please compile multiple times when
generating the references. The last entry in a reference are the back
references to the pages with the citation. They need an additional
compilation, once the bibtex entries are generated.

Note that the bibliography style is discipline dependent so feel free
to use the style adopted by your discipline, for example siam for
mathematics.

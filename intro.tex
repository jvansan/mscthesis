\chapter{Introduction}

This sample thesis\index{thesis} discusses changes from the
sample\index{sample} thesis of Michael Forbes, that make the thesis
compliant\index{compliant} with UBCO College of Graduate Studies
standards. If you need more information about the template and LaTeX,
please check out the sample thesis of Michael Forbes at

\href{http://alum.mit.edu/www/mforbes/projects/ubcthesis/}{http://alum.mit.edu/www/mforbes/projects/ubcthesis/}.

%Include citations in your thesis as you write:
\cite{MR2848848,MR2461448,MR2834159,infconv,convmono,MR2668638,Bauschke:2007-PA02,proxbas}

\section{Packages}

There are several packages\index{packages} included in
\texttt{ubcostyle.sty}. So before you add a new package, check first
if it is already included there.

\section{Epigraph}

If you want to add an epigraph to a chapter (epigraph in the sense of
a literary inscription, not a function epigraph), you can use the
command \texttt{epigraph} after the chapter. Check out the
documentation of the \texttt{epigraph} package for more information.

The following are examples of how to incorporate graphics into your thesis.

\begin{figure}[ht]
  \begin{center}
    \includegraphics[width=0.4\textwidth]{figure}
    \caption[Sample figure.]{\label{fig:happy} This is a sample figure
      Note that we have
      used the optional argument for the caption command so that only
      a short version of this caption occurs in the list of figures.}
  \end{center}
\end{figure}

\begin{figure}[ht]
  \begin{center}
    \includegraphics[width=0.4\textwidth]{figure}
    \caption{\label{fig:happy2} This is the same sample figure with still
			a long caption but this time we did not use a short caption command
			in the table of figures.}
  \end{center}
\end{figure}

You should really put text in between figures so LaTeX has more
flexibility to place the figure at the appropriate location.

\begin{figure}
	\centering

	\subfigure[Figure on the left side is identical to the one on the right.]{
		\includegraphics[width=150px]{figure}
		\label{fig:ex-ppa-l1-linf-1}
	}
	\subfigure[Figure on the right side is identical to the one on the left.]{
		\includegraphics[width=150px]{figure.pdf}
		\label{fig:ex-ppa-l1-linf-2}
	}

	\caption{An example of putting two figures side by side using the subfigure package.}
	\label{ref:ex-ppa-l1-linf}
\end{figure}

\begin{figure}%
\caption{Another Figure}%
\end{figure}

\begin{figure}%
\caption{Another Figure with a very long title to check the alignment in the lof}%
\end{figure}

\begin{figure}%
\caption{Another Figure}%
\end{figure}

\begin{figure}%
\caption{Another Figure}%
\end{figure}

\begin{figure}%
\caption{Another Figure}%
\end{figure}

\begin{figure}%
\caption{Another Figure}%
\end{figure}

\begin{figure}%
\caption{Another Figure}%
\end{figure}

\begin{table}
\caption{Short table title}
\end{table}
\begin{table}
\caption{Short table title}
\end{table}
\begin{table}
\caption{Long table title that wraps around several lines and goes on and on and on and on and on}
\end{table}
\begin{table}
\caption{Short table title}
\end{table}
\begin{table}
\caption{Short table title}
\end{table}
\begin{table}
\caption{Short table title}
\end{table}
\begin{table}
\caption{Short table title}
\end{table}
\begin{table}
\caption{Short table title}
\end{table}
